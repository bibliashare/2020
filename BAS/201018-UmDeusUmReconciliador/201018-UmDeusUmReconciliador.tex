%-----------------------------------------------------------------------------------------------
\documentclass[12pt,aspectratio=169]{beamer}
%-----------------------------------------------------------------------------------------------
\usepackage{pslatex}
%-----------------------------------------------------------------------------------------------
\newcommand{\YA}{%
    \mbox{%
        Y\makebox[0pt][l]{\hspace{-0.178em}\raisebox{-0.00ex}{\scalebox{0.30}{E}}}%
        H\makebox[0pt][l]{\hspace{-0.010em}\raisebox{-0.00ex}{\scalebox{0.30}{O}}}%
        W\makebox[0pt][l]{\hspace{-0.245em}\raisebox{-0.00ex}{\scalebox{0.30}{A}}}%
        H%
    }%
}
%-----------------------------------------------------------------------------------------------
\newcommand{\ver}[1]{%
    \raisebox{0.50ex}{%
        \scalebox{1.1}{%
            \pmb{\textbf{\textcolor{BSpbg}{#1}}}%
        }%
    }%
}
%-----------------------------------------------------------------------------------------------
\newcommand{\QUOTE}[1]{%
    \par\noindent\hspace*{0.1\linewidth}%
    \begin{minipage}{0.8\linewidth}%
        \linespread{1.35}\large{#1}%
    \end{minipage}%
}
%-----------------------------------------------------------------------------------------------
\newcommand{\RED}[1]{{\textcolor{TXred}{#1}}}
\newcommand{\YEL}[1]{{\textcolor{TXyel}{#1}}}
\newcommand{\GRE}[1]{{\textcolor{TXgre}{#1}}}
\newcommand{\CYA}[1]{{\textcolor{TXcya}{#1}}}
\newcommand{\BLU}[1]{{\textcolor{TXblu}{#1}}}
\newcommand{\MAG}[1]{{\textcolor{TXmag}{#1}}}
\newcommand{\BRI}[1]{{\textcolor{BSpbg}{#1}}}   % Bright
%-----------------------------------------------------------------------------------------------
\usetheme{CambridgeUS}
\usefonttheme{serif}
\usecolortheme{BShare1}
%-----------------------------------------------------------------------------------------------
\title{Um só Deus, um só Reconciliador}
\subtitle{1Tm 2.5}
\author{Bíblia Share}
%\institute{Bíblia Share}
\date[{\tiny\tt https://github.com/bibliashare}]{{\scriptsize\tt%
    \includegraphics[height=6.0mm]{res/cc/by-nc-nd-88x31.pdf}%\\[\smallskipamount]
    %https://github.com/bibliashare
}}
%-----------------------------------------------------------------------------------------------
\begin{document}
%-----------------------------------------------------------------------------------------------
\logo{%
    \parbox{158mm}{% There's a 1mm gap on each side of the 160mm x 90mm slide logo line
    \mode<beamer>{%
        \hfill\includegraphics[height=9.0mm]{res/logo/BibliaShare.pdf}%
    }
    \mode<handout>{%
        \hfill\includegraphics[height=9.0mm]{res/logo/BibliaShare.pdf}%
    }
}}
%-----------------------------------------------------------------------------------------------
\begin{frame}
    \titlepage
\end{frame}
%-----------------------------------------------------------------------------------------------
\section{Verso Base}
%-----------------------------------------------------------------------------------------------

    \begin{frame}{1 Timóteo 2.5 (Trad.~própria)}
        \QUOTE{%
            %-----!j 92 -i12
            \ver{5}~pois Deus {\it é}
            \alt<2,8>{\YEL{um}}{um} {\it só\/}, e {\it há}
            \alt<3,8>{\YEL{um}}{um} {\it só}
            \alt<4,8>{\MAG{reconciliador}}{reconciliador} entre
            \alt<5,8>{\BLU{Deus}}{Deus} e
            \alt<5,8>{\RED{homens}}{homens}, um
            \alt<6,8>{\MAG{homem}}{homem}---%
            \alt<7,8>{\MAG{Cristo Jesus}}{Cristo Jesus}.
        }
    \end{frame}

    \begin{frame}{1 Timóteo 2.5 -- Aplicações}
        \begin{itemize}
            \item<1-> Deus \YEL{existe}!
            \item<2-> Há \YEL{um só} Deus!
                \\[\bigskipamount]
            \item<3-> Existe \RED{necessidade de reconciliação} da humanidade com Deus!
                \\[\bigskipamount]
            \item<4-> Existe \MAG{reconciliador}!
            \item<5-> Há \YEL{um único} reconciliador!
                \uncover<6->{$\qquad\rightharpoondown\qquad$ \YEL{Um único} meio de
                reconciliação!}
                \\[\bigskipamount]
            \item<7-> O reconciliador entre \BLU{Deus} e \RED{homens} é um \MAG{homem}!
            \item<8-> O homem \MAG{Cristo Jesus}!
        \end{itemize}
    \end{frame}

%-----------------------------------------------------------------------------------------------
\section{Aprofundando}
%-----------------------------------------------------------------------------------------------

    \begin{frame}{João 1.1,14,18 (Trad.~própria)}
        \QUOTE{%
            %-----!j 92 -i12
            \normalsize
            \ver{1}~No princípio existia \MAG{a Palavra}, e \MAG{a Palavra}  estava  voltada
            para \BRI{Deus}, e \MAG{a Palavra era Deus}.  [...]  \ver{14}~e  \MAG{a  Palavra
            veio a ser carne} e armou tenda em nosso meio e contemplamos a sua \YEL{glória},
            uma \YEL{glória} como de unigênito do \BRI{Pai}, repleto  de  \YEL{graça}  e  de
            \YEL{verdade}.             [...]             \ver{18}~Ninguém             jamais
            \underline{v}iu\footnote{\textbf{\textsf{PERF}}:  \textit{ação  consumada  $+$  resultado
            permanecente\/}:  `viu  $+$  Deus  ficou  visto.'}  a  \BRI{Deus}:  o  \MAG{Deus
            unigênito}---que  permanece  estando  no  recôndito  do  \BRI{Pai}---este   {\it
            \BRI{o}} demonstrou.
        }
    \end{frame}

    \begin{frame}{João 1.1,14,18 -- Aplicações}
        \begin{itemize}
            \item<1-> A \MAG{Palavra}, que é \BLU{Deus}, também veio a ser \RED{carne} (homem);
            \item<2-> $\rightharpoondown\quad$ O homem \MAG{Cristo Jesus}!
            \item<3-> Jesus Cristo \YEL{é Deus}!$\qquad$\uncover<4->{\BRI{Ele é o meu Deus!}}
            \item<5-> Jesus Cristo \YEL{é} o \BLU{Deus Filho}, que veio a ser \RED{homem}.
            \item<6-> O apóstolo que testemunha viu sua \YEL{glória}, \YEL{graça} e
                \YEL{verdade}! \\[\bigskipamount]
            \item<7-> Assim, em 1Tm~2.5, o reconciliador entre \BLU{Deus} e \RED{homens} é um
                \MAG{homem}!
            \item<8-> E em Jo~1.1,14,18, vemos que tal \MAG{Palavra} \RED{encarnada} era
                \BLU{Deus} desde o princípio! \\[\bigskipamount]
            \item<9-> \MAG{Cristo Jesus}: o \YEL{único} e perfeito \MAG{reconciliador} entre
                \BLU{Deus} e \RED{homens}!
        \end{itemize}
    \end{frame}

%-----------------------------------------------------------------------------------------------
\end{document}
%-----------------------------------------------------------------------------------------------

%-----------------------------------------------------------------------------------------------
\documentclass[12pt,aspectratio=169]{beamer}
%-----------------------------------------------------------------------------------------------
\usepackage{pslatex}
%-----------------------------------------------------------------------------------------------
\newcommand{\YA}{%
    \mbox{%
        Y\makebox[0pt][l]{\hspace{-0.178em}\raisebox{-0.00ex}{\scalebox{0.30}{E}}}%
        H\makebox[0pt][l]{\hspace{-0.010em}\raisebox{-0.00ex}{\scalebox{0.30}{O}}}%
        W\makebox[0pt][l]{\hspace{-0.245em}\raisebox{-0.00ex}{\scalebox{0.30}{A}}}%
        H%
    }%
}
%-----------------------------------------------------------------------------------------------
\newcommand{\ver}[1]{%
    \raisebox{0.50ex}{%
        \scalebox{1.1}{%
            \pmb{\textbf{\textcolor{BSpbg}{#1}}}%
        }%
    }%
}
%-----------------------------------------------------------------------------------------------
\newcommand{\QUOTE}[1]{%
    \par\noindent\hspace*{0.1\linewidth}%
    \begin{minipage}{0.8\linewidth}%
        \linespread{1.35}\large{#1}%
    \end{minipage}%
}
%-----------------------------------------------------------------------------------------------
\newcommand{\RED}[1]{{\textcolor{TXred}{#1}}}
\newcommand{\YEL}[1]{{\textcolor{TXyel}{#1}}}
\newcommand{\GRE}[1]{{\textcolor{TXgre}{#1}}}
\newcommand{\CYA}[1]{{\textcolor{TXcya}{#1}}}
\newcommand{\BLU}[1]{{\textcolor{TXblu}{#1}}}
\newcommand{\MAG}[1]{{\textcolor{TXmag}{#1}}}
\newcommand{\BRI}[1]{{\textcolor{BSpbg}{#1}}}   % Bright
%-----------------------------------------------------------------------------------------------
\usetheme{CambridgeUS}
\usefonttheme{serif}
\usecolortheme{BShare1}
%-----------------------------------------------------------------------------------------------
\title{Vida Abundante em Cristo}
\subtitle{Parte 1 de 2}
\author{Bíblia Share}
%\institute{Bíblia Share}
\date[{\tiny\tt\today}]{{\scriptsize\tt%
    \includegraphics[height=6.0mm]{res/cc/by-nc-nd-88x31.pdf}\\[\smallskipamount]
    \today
}}
%-----------------------------------------------------------------------------------------------
\begin{document}
%-----------------------------------------------------------------------------------------------
\logo{%
    \parbox{158mm}{% There's a 1mm gap on each side of the 160mm x 90mm slide logo line
    \mode<beamer>{%
        \hfill\includegraphics[height=9.0mm]{res/logo/BibliaShare.pdf}%
    }
    \mode<handout>{%
        \hfill\includegraphics[height=9.0mm]{res/logo/BibliaShare.pdf}%
    }
}}
%-----------------------------------------------------------------------------------------------
\begin{frame}
    \titlepage
\end{frame}
%-----------------------------------------------------------------------------------------------
\section{Introdução}
%-----------------------------------------------------------------------------------------------

    \begin{frame}{\BRI{Vida Abundante} \YEL{\emph{em} Cristo} -- Divisão}
        \begin{itemize}
            \item<1-> \BRI{Parte 1} -- Base de verdade:\\
                \uncover<2->{Nossa \YEL{posição} \MAG{em Jesus Cristo};}
                \\[\bigskipamount]
            \item<3-> \BRI{Parte 2} -- Aplicação:\\
                \uncover<4->{A \CYA{vida abundante} decorrente da nossa \YEL{identificação}
                \MAG{com Cristo}.}
        \end{itemize}
    \end{frame}

    \begin{frame}{Tópicos da oração}
        \begin{enumerate}
            \item<2-> \YEL{Graças} por estar \RED{morto para o pecado} e \GRE{vivo para Deus} \MAG{em Cristo Jesus};
                \\[\smallskipamount]
            \item<3-> \YEL{Escolho} não mais viver pelo \RED{meu ponto de vista};
            \item<4-> \YEL{Escolho} não entregar ao \RED{meu ponto de vista} nenhum \BRI{membro} do corpo;
            \item<5-> \YEL{Apresento-me} a Deus como pessoa \GRE{ressurreta} de entre os mortos;
            \item<6-> \YEL{Apresento} meus \BRI{membros} a Deus como \RED{instrumentos de Sua justiça}.
        \end{enumerate}
    \end{frame}

    %-------------------------------------------------------------------------------------------
    \subsection{2Co 5.14-17}
    %-------------------------------------------------------------------------------------------

    \begin{frame}{2Co 5.14-17 (ARA)}
        \QUOTE{%
            \normalsize
            %-----!j 92 -i12
            \ver{14}~Pois o amor de Cristo nos constrange,
            julgando nós isto:
            um morreu por todos; logo,
            \alt<2->{\YEL{todos morreram}}{todos morreram}.
            %-----!j 92 -i12
            \ver{15}~E ele morreu por todos,
            para que os que vivem não vivam mais
            \alt<3->{\YEL{para si mesmos}}{para si mesmos},
            mas para aquele que por eles morreu e ressuscitou.
            %-----!j 92 -i12
            \ver{16}~Assim que, nós, daqui por diante,
            a ninguém conhecemos segundo a carne;
            e, se antes conhecemos \alt<4->{\MAG{Cristo}}{Cristo} segundo a carne,
            já agora \alt<5->{\YEL{não o conhecemos deste modo}}{não o conhecemos deste modo}.
            %-----!j 92 -i12
            \ver{17}~E, assim,
            se alguém \alt<6->{\MAG{está em Cristo}}{está em Cristo},
            é \alt<7->{\GRE{nova criatura}}{nova criatura};
            as coisas antigas já passaram;
            eis que se fizeram novas.
        }
    \end{frame}

    %-------------------------------------------------------------------------------------------
    \subsection{Ef 2.10}
    %-------------------------------------------------------------------------------------------

    \begin{frame}{Ef 2.10 (ARA)}
        \QUOTE{%
            %-----!j 92 -i12
            \ver{10}~Pois somos \alt<2->{\YEL{feitura dele}}{feitura dele},
            criados \alt<3->{\MAG{em Cristo Jesus}}{em Cristo Jesus}
            para \alt<6->{\BRI{boas obras}}{boas obras},
            as quais Deus de \alt<4->{\YEL{antemão preparou}}{antemão preparou}
            para que \alt<5->{\GRE{andássemos}}{andássemos} nelas.
        }
    \end{frame}

    \begin{frame}{Jr 10.23 (ARA)}
        \QUOTE{%
            %-----!j 92 -i12
            \ver{23}~Eu sei, ó Senhor,
            que não cabe ao homem
            \alt<2->{\YEL{determinar}}{determinar} o seu caminho,
            nem ao que \alt<3->{\GRE{caminha}}{caminha}
            o \alt<4->{\YEL{dirigir}}{dirigir} os seus passos.
        }
    \end{frame}

    %-------------------------------------------------------------------------------------------
    \subsection{Jo 10.10}
    %-------------------------------------------------------------------------------------------

    \begin{frame}{Jo 10.10 (ARA)}
        \QUOTE{%
            %-----!j 92 -i12
            \ver{10}~O ladrão vem somente para roubar, matar e destruir;
            eu vim para que tenham \alt<2->{\YEL{vida}}{vida}
            e a tenham em \alt<3->{\GRE{abundância}}{abundância}.
        }
    \end{frame}

    \begin{frame}{Jo 3.20,21 (ARA)}
        \QUOTE{%
            %-----!j 92 -i12
            \ver{20}~Pois todo aquele que \alt<2->{\RED{pratica o mal}}{pratica o mal}
            aborrece a luz e não se chega para a luz,
            a fim de não serem arguidas as suas obras.
            \ver{21}~Quem \alt<3->{\GRE{pratica a \emph{verdade}}}{pratica a verdade}
            aproxima-se da luz,
            a fim de que as suas obras sejam manifestas,
            porque \alt<4->{\YEL{feitas} \MAG{em Deus}}{feitas em Deus}.
        }
    \end{frame}

%-----------------------------------------------------------------------------------------------
\section{Jesus é o Último Adão}
%-----------------------------------------------------------------------------------------------

    %-------------------------------------------------------------------------------------------
    \subsection{1Co 15.45-49}
    %-------------------------------------------------------------------------------------------

    \begin{frame}{1Co 15.45-49 (ARA)}
        \QUOTE{%
            \normalsize
            %-----!j 92 -i12
            \ver{45}~Pois assim está escrito:
            O primeiro homem, Adão, foi feito alma vivente.
            O \alt<2->{\MAG{último Adão}}{último Adão}, porém, é espírito vivificante.
            %-----!j 92 -i12
            \ver{46}~Mas não é primeiro o espiritual,
            e sim o \alt<3->{\YEL{natural}}{natural};
            depois, o \alt<4->{\YEL{espiritual}}{espiritual}.
            %-----!j 92 -i12
            \ver{47}~O primeiro homem,
            formado da terra, \alt<5->{\RED{é terreno}}{é terreno};
            o segundo homem \alt<6->{\GRE{é do céu}}{é do céu}.
            %-----!j 92 -i12
            \ver{48}~Como foi o primeiro homem, o terreno,
            \alt<7->{\YEL{tais são também}}{tais são também} os demais homens terrenos;
            e, como é o homem celestial,
            \alt<8->{\YEL{tais também}}{tais também} os celestiais.
            %-----!j 92 -i12
            \ver{49}~E, assim como trouxemos a imagem do que é terreno,
            \alt<9->{\YEL{devemos trazer}}{devemos trazer}
            \only<10>{(= \YEL{traremos})}
            também a imagem do celestial.
        }
    \end{frame}

    %-------------------------------------------------------------------------------------------
    \subsection{Rm 5.14}
    %-------------------------------------------------------------------------------------------

    \begin{frame}{Rm 5.14 (ARA)}
        \QUOTE{%
            %-----!j 92 -i12
            \ver{14}~Entretanto, reinou a morte desde Adão até Moisés,
            mesmo sobre aqueles que não pecaram à semelhança da transgressão de Adão,
            o qual prefigurava aquele que havia de vir.
        }
    \end{frame}

%-----------------------------------------------------------------------------------------------
\end{document}
%-----------------------------------------------------------------------------------------------

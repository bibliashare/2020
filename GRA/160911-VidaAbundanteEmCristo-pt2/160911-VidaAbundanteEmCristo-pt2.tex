%-----------------------------------------------------------------------------------------------
\documentclass[12pt,aspectratio=169]{beamer}
%-----------------------------------------------------------------------------------------------
\usepackage{pslatex}
%-----------------------------------------------------------------------------------------------
\newcommand{\YA}{%
    \mbox{%
        Y\makebox[0pt][l]{\hspace{-0.178em}\raisebox{-0.00ex}{\scalebox{0.30}{E}}}%
        H\makebox[0pt][l]{\hspace{-0.010em}\raisebox{-0.00ex}{\scalebox{0.30}{O}}}%
        W\makebox[0pt][l]{\hspace{-0.245em}\raisebox{-0.00ex}{\scalebox{0.30}{A}}}%
        H%
    }%
}
%-----------------------------------------------------------------------------------------------
\newcommand{\ver}[1]{%
    \raisebox{0.50ex}{%
        \scalebox{1.1}{%
            \pmb{\textbf{\textcolor{BSpbg}{#1}}}%
        }%
    }%
}
%-----------------------------------------------------------------------------------------------
\newcommand{\QUOTE}[1]{%
    \par\noindent\hspace*{0.05\linewidth}%
    \begin{minipage}{0.9\linewidth}%
        \linespread{1.35}\large{#1}%
    \end{minipage}%
}
%-----------------------------------------------------------------------------------------------
\newcommand{\RED}[1]{{\textcolor{TXred}{#1}}}
\newcommand{\YEL}[1]{{\textcolor{TXyel}{#1}}}
\newcommand{\GRE}[1]{{\textcolor{TXgre}{#1}}}
\newcommand{\CYA}[1]{{\textcolor{TXcya}{#1}}}
\newcommand{\BLU}[1]{{\textcolor{TXblu}{#1}}}
\newcommand{\MAG}[1]{{\textcolor{TXmag}{#1}}}
\newcommand{\BRI}[1]{{\textcolor{BSpbg}{#1}}}   % Bright
%-----------------------------------------------------------------------------------------------
\usetheme{CambridgeUS}
\usefonttheme{serif}
\usecolortheme{BShare1}
%-----------------------------------------------------------------------------------------------
\title{Vida Abundante em Cristo}
\subtitle{Parte 2 de 2}
\author{Bíblia Share}
%\institute{Bíblia Share}
\date[{\tiny\tt 11 de Setembro de 2016}]{{\scriptsize\tt%
    \includegraphics[height=6.0mm]{res/cc/by-nc-nd-88x31.pdf}\\[\smallskipamount]
    11 de Setembro de 2016
}}
%-----------------------------------------------------------------------------------------------
\begin{document}
%-----------------------------------------------------------------------------------------------
\logo{%
    \parbox{158mm}{% There's a 1mm gap on each side of the 160mm x 90mm slide logo line
    \mode<beamer>{%
        \hfill\includegraphics[height=9.0mm]{res/logo/BibliaShare.pdf}%
    }
    \mode<handout>{%
        \hfill\includegraphics[height=9.0mm]{res/logo/BibliaShare.pdf}%
    }
}}
%-----------------------------------------------------------------------------------------------
\begin{frame}
    \titlepage
\end{frame}
%-----------------------------------------------------------------------------------------------
\section{Introdução}
%-----------------------------------------------------------------------------------------------

    \begin{frame}{\BRI{Vida Abundante} \YEL{\emph{em} Cristo} -- Divisão}
        \begin{itemize}
            \item \BRI{Parte 1} -- Base de verdade:\\
                Nossa \YEL{posição} \MAG{em Jesus Cristo};
                \\[\bigskipamount]
            \item \BRI{Parte 2} -- Aplicação:\\
                A \CYA{vida abundante} decorrente da nossa \YEL{identificação}
                \MAG{com Cristo}.
        \end{itemize}
    \end{frame}

%-----------------------------------------------------------------------------------------------
\section{Recapitulação da Parte 1}
%-----------------------------------------------------------------------------------------------

    %-------------------------------------------------------------------------------------------
    \subsection{2Co 5.14--17 -- Novas criaturas em Cristo Jesus}
    %-------------------------------------------------------------------------------------------

    \begin{frame}{2Co 5.17 (ARC) -- \YEL{Novas criaturas} \MAG{em Cristo Jesus}}
        \QUOTE{%
            %-----!j 92 -i12
            \ver{17}~Assim que, se alguém \MAG{está em Cristo}, \GRE{nova criatura} é: as coisas
            velhas já passaram; eis que tudo se fez novo.
        }
    \end{frame}

    \begin{frame}{2Co 5.15 (ARC) -- Não viver mais \YEL{\emph{de} si mesmo}}
        \QUOTE{%
            %-----!j 92 -i12
            \ver{15}~E ele morreu por todos, para que os que vivem \YEL{não vivam mais para si},
            mas para aquele que por eles morreu e ressuscitou.
        }
    \end{frame}

    %-------------------------------------------------------------------------------------------
    \subsection{1Co 15.45--49 -- Jesus Cristo é o Último Adão}
    %-------------------------------------------------------------------------------------------

    \begin{frame}{1Co 15.45,48 (ARA) -- \YEL{Último Adão}; homens \RED{terrenos} e \GRE{do céu}}
        \QUOTE{%
            %-----!j 92 -i12
            \ver{45}~Assim está também escrito: O \RED{primeiro homem, Adão}, foi feito em alma
            vivente; o \GRE{último Adão}, em espírito vivificante.
            %-----!j 92 -i12
            \ver{48}~Qual o \RED{terreno}, \YEL{tais são também} os terrenos; e, qual o
            \GRE{celestial}, \YEL{tais também} os celestiais.
        }
    \end{frame}

    %-------------------------------------------------------------------------------------------
    \subsection{Rm 5.12-14 -- Herança do Primeiro Adão}
    %-------------------------------------------------------------------------------------------

    \begin{frame}{Rm 5.12,14 (ARA) -- Herança de \RED{pecado} e \RED{morte}}
        \QUOTE{%
            %-----!j 92 -i12
            \ver{12}~Pelo que, como \YEL{por um homem} entrou o \RED{pecado} no mundo, e pelo
            pecado, a \RED{morte}, assim também a morte passou a todos os homens, por isso que
            todos pecaram.
            %-----!j 92 -i12
            \ver{14}~No entanto, a morte reinou desde Adão até Moisés, até sobre aqueles que não
            pecaram à semelhança da transgressão de \YEL{Adão}, o qual é a \YEL{figura daquele
            que havia de vir}.
        }
    \end{frame}

    \begin{frame}{Tipo de Pecadores: \RED{Atitude de ``sabe-tudo''}}
        \QUOTE{%
            %-----!j 92 -i12
            \ver{Gn 2.9 (ARA)}~[...] a árvore da vida no meio do jardim e a \RED{árvore do
            conhecimento do bem} (do que é BOM / \YEL{VANTAJOSO}) \RED{e do mal} (do que é MAU /
            \YEL{DESVANTAJOSO}).
            \\[\bigskipamount]
            %-----!j 92 -i12
            \ver{Rm 1.22 (ARC)}~\RED{Dizendo-se} sábios, tornaram-se loucos.
        }
    \end{frame}

    %-------------------------------------------------------------------------------------------
    \subsection{Rm 6.3--5 -- Herança do Último Adão}
    %-------------------------------------------------------------------------------------------

    \begin{frame}{Rm 6.6,4 (ARC) -- \YEL{Crucificação, Sepultamento e Ressurreição}}
        \QUOTE{%
            %-----!j 92 -i12
            \ver{6}~sabendo isto: que o nosso velho homem foi com ele \YEL{crucificado}, para
            que o corpo do pecado seja desfeito, a fim de que não sirvamos mais ao pecado.
            %-----!j 92 -i12
            \ver{4}~De sorte que fomos \YEL{sepultados} com ele pelo batismo na morte; para que,
            como Cristo \YEL{ressuscitou} dos mortos pela glória do Pai, assim andemos nós
            também em novidade de vida.
        }
    \end{frame}

    \begin{frame}{\YEL{Crucificação, Sepultamento e Ressurreição}}
        \begin{itemize}
            \item \YEL{Crucificação} significa \YEL{4 coisas};
                \\[\bigskipamount]
            \item \YEL{Sepultamento} significa \YEL{2 coisas};
                \\[\bigskipamount]
            \item \YEL{Ressurreição} significa ao menos \YEL{3 coisas}.
        \end{itemize}
    \end{frame}

    %-------------------------------------------------------------------------------------------
    \subsection{Crucificação}
    %-------------------------------------------------------------------------------------------

    \begin{frame}{1. \YEL{Mortos para o pecado} -- Rm 6.2,10 (ARC)}
        \QUOTE{%
            %-----!j 92 -i12
            \ver{2}~De modo nenhum! \YEL{Nós} que \YEL{estamos mortos para o pecado}, como
            viveremos ainda nele?
            %-----!j 92 -i12
            \ver{10}~Pois, quanto a ter morrido, de uma vez \YEL{morreu para o pecado}; mas,
            quanto a viver, vive para Deus.
        }
        \\[2\bigskipamount]
        \begin{itemize}
            \item<2-> Mortos para a atitude de ``sabe-tudo'';
            \item<3-> Podemos deixar de ser iniciadores dos nossos atos.
        \end{itemize}
    \end{frame}

    \begin{frame}{2. \YEL{Velho homem crucificado}}
        \QUOTE{%
            %-----!j 92 -i12
            \ver{Rm 6.6 (ARC)}~sabendo isto: que o nosso velho homem \YEL{foi com ele
            crucificado}, para que o corpo do pecado seja desfeito, a fim de que não sirvamos
            mais ao pecado.
            \\[\bigskipamount]
            %-----!j 92 -i12
            \ver{Gl 5.24 (ARC)}~E os que são de Cristo \YEL{crucificaram a carne} com as suas
            paixões e concupiscências.
        }
        \\[2\bigskipamount]
        \begin{itemize}
            \item<2-> Mortos para a atitude de ``sabe-tudo'';
            \item<3-> Podemos deixar de ser iniciadores dos nossos atos.
        \end{itemize}
    \end{frame}

%-----------------------------------------------------------------------------------------------
\end{document}
%-----------------------------------------------------------------------------------------------

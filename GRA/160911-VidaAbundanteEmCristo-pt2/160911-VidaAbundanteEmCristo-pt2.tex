%-----------------------------------------------------------------------------------------------
\documentclass[12pt,aspectratio=169]{beamer}
%-----------------------------------------------------------------------------------------------
\usepackage{pslatex}
%-----------------------------------------------------------------------------------------------
\newcommand{\YA}{%
    \mbox{%
        Y\makebox[0pt][l]{\hspace{-0.178em}\raisebox{-0.00ex}{\scalebox{0.30}{E}}}%
        H\makebox[0pt][l]{\hspace{-0.010em}\raisebox{-0.00ex}{\scalebox{0.30}{O}}}%
        W\makebox[0pt][l]{\hspace{-0.245em}\raisebox{-0.00ex}{\scalebox{0.30}{A}}}%
        H%
    }%
}
%-----------------------------------------------------------------------------------------------
\newcommand{\ver}[1]{%
    \raisebox{0.50ex}{%
        \scalebox{1.1}{%
            \pmb{\textbf{\textcolor{BSpbg}{#1}}}%
        }%
    }%
}
%-----------------------------------------------------------------------------------------------
\newcommand{\QUOTE}[1]{%
    \par\noindent\hspace*{0.05\linewidth}%
    \begin{minipage}{0.9\linewidth}%
        \linespread{1.35}\large{#1}%
    \end{minipage}%
}
%-----------------------------------------------------------------------------------------------
\newcommand{\RED}[1]{{\textcolor{TXred}{#1}}}
\newcommand{\ORA}[1]{{\textcolor{TXred!50!TXyel}{#1}}}
\newcommand{\YEL}[1]{{\textcolor{TXyel}{#1}}}
\newcommand{\GRE}[1]{{\textcolor{TXgre}{#1}}}
\newcommand{\CYA}[1]{{\textcolor{TXcya}{#1}}}
\newcommand{\BLU}[1]{{\textcolor{TXblu}{#1}}}
\newcommand{\MAG}[1]{{\textcolor{TXmag}{#1}}}
\newcommand{\BRI}[1]{{\textcolor{BSpbg}{#1}}}   % Bright
%-----------------------------------------------------------------------------------------------
\usetheme{CambridgeUS}
\usefonttheme{serif}
\usecolortheme{BShare1}
%-----------------------------------------------------------------------------------------------
\title{Vida Abundante em Cristo}
\subtitle{Parte 2 de 2}
\author{Bíblia Share}
%\institute{Bíblia Share}
\date[{\tiny\tt 11 de Setembro de 2016}]{{\scriptsize\tt%
    \includegraphics[height=6.0mm]{res/cc/by-nc-nd-88x31.pdf}\\[\smallskipamount]
    11 de Setembro de 2016
}}
%-----------------------------------------------------------------------------------------------
\begin{document}
%-----------------------------------------------------------------------------------------------
\logo{%
    \parbox{158mm}{% There's a 1mm gap on each side of the 160mm x 90mm slide logo line
    \mode<beamer>{%
        \hfill\includegraphics[height=9.0mm]{res/logo/BibliaShare.pdf}%
    }
    \mode<handout>{%
        \hfill\includegraphics[height=9.0mm]{res/logo/BibliaShare.pdf}%
    }
}}
%-----------------------------------------------------------------------------------------------
\begin{frame}
    \titlepage
\end{frame}
%-----------------------------------------------------------------------------------------------
\section{Introdução}
%-----------------------------------------------------------------------------------------------

    \begin{frame}{\BRI{Vida Abundante} \YEL{\emph{em} Cristo} -- Divisão}
        \begin{itemize}
            \item \BRI{Parte 1} -- Base de verdade:\\
                Nossa \YEL{posição} \MAG{em Jesus Cristo};
                \\[\bigskipamount]
            \item \BRI{Parte 2} -- Aplicação:\\
                A \CYA{vida abundante} decorrente da nossa \YEL{identificação}
                \MAG{com Cristo}.
        \end{itemize}
    \end{frame}

%-----------------------------------------------------------------------------------------------
\section{Recapitulação da Parte 1}
%-----------------------------------------------------------------------------------------------

    %-------------------------------------------------------------------------------------------
    \subsection{2Co 5.14--17 -- Novas criaturas em Cristo Jesus}
    %-------------------------------------------------------------------------------------------

    \begin{frame}{2Co 5.17 (ARC) -- \YEL{Novas criaturas} \MAG{em Cristo Jesus}}
        \QUOTE{%
            %-----!j 92 -i12
            \ver{17}~Assim que, se alguém \MAG{está em Cristo}, \GRE{nova criatura} é: as coisas
            velhas já passaram; eis que tudo se fez novo.
        }
    \end{frame}

    \begin{frame}{2Co 5.15 (ARC) -- Não viver mais \YEL{\emph{de} si mesmo}}
        \QUOTE{%
            %-----!j 92 -i12
            \ver{15}~E ele morreu por todos, para que os que vivem \YEL{não vivam mais para si},
            mas para aquele que por eles morreu e ressuscitou.
        }
    \end{frame}

    %-------------------------------------------------------------------------------------------
    \subsection{1Co 15.45--49 -- Jesus Cristo é o Último Adão}
    %-------------------------------------------------------------------------------------------

    \begin{frame}{1Co 15.45,48 (ARA) -- \YEL{Último Adão}; homens \RED{terrenos} e \GRE{do céu}}
        \QUOTE{%
            %-----!j 92 -i12
            \ver{45}~Assim está também escrito: O \RED{primeiro homem, Adão}, foi feito em alma
            vivente; o \GRE{último Adão}, em espírito vivificante.
            %-----!j 92 -i12
            \ver{48}~Qual o \RED{terreno}, \YEL{tais são também} os terrenos; e, qual o
            \GRE{celestial}, \YEL{tais também} os celestiais.
        }
    \end{frame}

    %-------------------------------------------------------------------------------------------
    \subsection{Rm 5.12-14 -- Herança do Primeiro Adão}
    %-------------------------------------------------------------------------------------------

    \begin{frame}{Rm 5.12,14 (ARA) -- Herança de \RED{pecado} e \RED{morte}}
        \QUOTE{%
            %-----!j 92 -i12
            \ver{12}~Pelo que, como \YEL{por um homem} entrou o \RED{pecado} no mundo, e pelo
            pecado, a \RED{morte}, assim também a morte passou a todos os homens, por isso que
            todos pecaram.
            %-----!j 92 -i12
            \ver{14}~No entanto, a morte reinou desde Adão até Moisés, até sobre aqueles que não
            pecaram à semelhança da transgressão de \YEL{Adão}, o qual é a \YEL{figura daquele
            que havia de vir}.
        }
    \end{frame}

    \begin{frame}{Tipo de Pecadores: \RED{Atitude de ``sabe-tudo''}}
        \QUOTE{%
            %-----!j 92 -i12
            \ver{Gn 2.9 (ARA)}~[...] a árvore da vida no meio do jardim e a \RED{árvore do
            conhecimento do bem} (do que é BOM / \YEL{VANTAJOSO}) \RED{e do mal} (do que é MAU /
            \YEL{DESVANTAJOSO}).
            \\[\bigskipamount]
            %-----!j 92 -i12
            \ver{Rm 1.22 (ARC)}~\RED{Dizendo-se} sábios, tornaram-se loucos.
        }
    \end{frame}

    %-------------------------------------------------------------------------------------------
    \subsection{Rm 6.3--5 -- Herança do Último Adão}
    %-------------------------------------------------------------------------------------------

    \begin{frame}{Rm 6.6,4 (ARC) -- \YEL{Crucificação, Sepultamento e Ressurreição}}
        \QUOTE{%
            %-----!j 92 -i12
            \ver{6}~sabendo isto: que o nosso velho homem foi com ele \YEL{crucificado}, para
            que o corpo do pecado seja desfeito, a fim de que não sirvamos mais ao pecado.
            %-----!j 92 -i12
            \ver{4}~De sorte que fomos \YEL{sepultados} com ele pelo batismo na morte; para que,
            como Cristo \YEL{ressuscitou} dos mortos pela glória do Pai, assim andemos nós
            também em novidade de vida.
        }
    \end{frame}

    \begin{frame}{\YEL{Crucificação, Sepultamento e Ressurreição}}
        \begin{itemize}
            \item \YEL{Crucificação} significa \YEL{4 coisas};
                \\[\bigskipamount]
            \item \YEL{Sepultamento} significa \YEL{2 coisas};
                \\[\bigskipamount]
            \item \YEL{Ressurreição} significa ao menos \YEL{3 coisas}.
        \end{itemize}
    \end{frame}

    %-------------------------------------------------------------------------------------------
    \subsection{Crucificação}
    %-------------------------------------------------------------------------------------------

    \begin{frame}{1. \YEL{Mortos para o pecado} -- Rm 6.2,10 (ARC)}
        \QUOTE{%
            %-----!j 92 -i12
            \ver{2}~De modo nenhum! \YEL{Nós} que \YEL{estamos mortos para o pecado}, como
            viveremos ainda nele?
            %-----!j 92 -i12
            \ver{10}~Pois, quanto a ter morrido, de uma vez \YEL{morreu para o pecado}; mas,
            quanto a viver, vive para Deus.
        }
        \\[1.5\bigskipamount]
        \begin{itemize}
            \item<2-> Mortos para a atitude de ``sabe-tudo'';
            \item<3-> Podemos deixar de ser iniciadores dos nossos atos.
        \end{itemize}
    \end{frame}

    \begin{frame}{2. \YEL{Velho homem crucificado}}
        \QUOTE{%
            %-----!j 92 -i12
            \ver{Rm 6.6 (ARC)}~sabendo isto: que o nosso velho homem \YEL{foi com ele
            crucificado}, para que o corpo do pecado seja desfeito, a fim de que não sirvamos
            mais ao pecado.
            \\[\bigskipamount]
            %-----!j 92 -i12
            \ver{Gl 5.24 (ARC)}~E os que são de Cristo \YEL{crucificaram a carne} com as suas
            paixões e concupiscências.
        }
        \\[1.5\bigskipamount]
        \begin{itemize}
            \item<2-> Velho homem: o \BRI{tipo de pessoa} que somos após termos nascido na raça
                de Adão.
            \item<3-> Diferença entre \BRI{problema} de pecado (todos) e \BRI{padrão} de pecados
                (indivíduo).
        \end{itemize}
    \end{frame}

    \begin{frame}{3. \YEL{Mortos para o mundo} -- Gl 6.14 (ARC)}
        \QUOTE{%
            %-----!j 92 -i12
            \ver{14}~Mas longe esteja de mim gloriar-me, a não ser na cruz de nosso Senhor Jesus
            Cristo, pela qual \YEL{o mundo está crucificado para mim e eu, para o mundo}.
        }
        \\[1.5\bigskipamount]
        \begin{itemize}
            \item<2-> Morremos para o estar vivendo uma vida \BRI{mundana}:
            \item<3-> Morremos para o não \BRI{andar com Deus} / não estar \BRI{cheio do Espírito}.
        \end{itemize}
    \end{frame}

    \begin{frame}{4. \YEL{Mortos para a lei} -- Rm 7.4 (ARC)}
        \QUOTE{%
            %-----!j 92 -i12
            \ver{4}~Assim, meus irmãos, também vós \YEL{estais mortos para a lei} pelo
            \MAG{corpo de Cristo}, para que sejais doutro, daquele que ressuscitou de entre os
            mortos, a fim de que \GRE{demos fruto para Deus}.
        }
        \\[1.5\bigskipamount]
        \begin{itemize}
            \item<2-> Povo sob a lei: ``\BRI{tudo o que o Senhor falou, faremos}'' -- Êx 19.8; 24.3,7;
            \item<3-> Sob lei: Responsáveis pelos \BRI{próprios atos} / pela \BRI{vida!}
            \item<4-> Morremos para estratégias como: \BRI{fazer o \emph{meu} melhor para Deus!}
        \end{itemize}
    \end{frame}

    \begin{frame}{4. \YEL{Mortos para a lei} -- Rm 8.3,4 (ARA)}
        \QUOTE{%
            %-----!j 92 -i12
            \ver{3}~Porquanto, o que era \YEL{impossível} à lei, visto como estava \RED{enferma
            pela carne}, Deus, enviando o seu Filho [...]
            %-----!j 92 -i12
            \ver{4}~para que a justiça da lei se cumprisse em nós, que não andamos segundo a
            \RED{carne}, mas segundo o \GRE{Espírito}.
        }
    \end{frame}

    \begin{frame}{4. \YEL{Mortos para a lei} -- Rm 7.6 (ARC)}
        \QUOTE{%
            %-----!j 92 -i12
            \ver{6}~Mas, agora, estamos \YEL{livres da lei}, pois \YEL{morremos} para aquilo em
            que estávamos retidos; para que \GRE{sirvamos em novidade de espírito}, e não na
            \RED{velhice da letra}.
        }
        \\[1.5\bigskipamount]
        \begin{itemize}
            \item<2-> Porta para deixarmos de servir a Deus \BRI{pelas nossas forças};
            \item<3-> Ou \BRI{pelos nossos méritos}.
        \end{itemize}
    \end{frame}

    %-------------------------------------------------------------------------------------------
    \subsection{Sepultamento}
    %-------------------------------------------------------------------------------------------

    \begin{frame}{\YEL{Sepultamento} -- Rm 6.4 (ARC)}
        \QUOTE{%
            %-----!j 92 -i12
            \ver{4}~De sorte que fomos \YEL{sepultados} com ele pelo batismo na morte; para que,
            como Cristo \YEL{ressuscitou} dos mortos pela glória do Pai, assim andemos nós
            também em novidade de vida.
        }
        \\[1.5\bigskipamount]
        \begin{enumerate}
            \item<2-> Velho homem também sepultado: uma ``\BRI{segunda volta na chave}'';
            \item<3-> Uma preparação para a \BRI{ressurreição}.
        \end{enumerate}
    \end{frame}

    %-------------------------------------------------------------------------------------------
    \subsection{Ressurreição}
    %-------------------------------------------------------------------------------------------

    \begin{frame}{1. \YEL{Recebemos a vida de Deus} -- Ef 2.5,6 (ARA)}
        \QUOTE{%
            %-----!j 92 -i12
            \ver{5}~e estando nós mortos em nossos delitos, \YEL{nos deu vida juntamente}
            \MAG{com Cristo}, --- pela graça sois salvos,
            %-----!j 92 -i12
            \ver{6}~e, \YEL{juntamente com ele, nos ressuscitou}, e \CYA{nos fez assentar nos
            lugares celestiais} \MAG{em Cristo Jesus};
        }
        \\[1.5\bigskipamount]
        \begin{enumerate}
            \item<2-> Vida de Deus voltou para o corpo de Jesus crucificado e sepultado.
        \end{enumerate}
    \end{frame}

    \begin{frame}{2. \YEL{Fomos levantados-junto} -- Ef 2.5,6 (ARA)}
        \QUOTE{%
            %-----!j 92 -i12
            \ver{5}~e estando nós mortos em nossos delitos, \YEL{nos deu vida juntamente}
            \MAG{com Cristo}, --- pela graça sois salvos,
            %-----!j 92 -i12
            \ver{6}~e, \YEL{juntamente com ele, nos ressuscitou}, e \CYA{nos fez assentar nos
            lugares celestiais} \MAG{em Cristo Jesus};
        }
        \\[1.5\bigskipamount]
        \begin{enumerate}
            \item<2-> Ressurretos de entre os mortos, deixamos o lugar dos mortos;
            \item<3-> Testemunhado na mudança de relacionamentos após conversão em adultos.
        \end{enumerate}
    \end{frame}

    \begin{frame}{3. \YEL{Assentados-junto em Cristo nos lugares celestiais} -- Ef 2.5,6}
        \QUOTE{%
            %-----!j 92 -i12
            \ver{5}~e estando nós mortos em nossos delitos, \YEL{nos deu vida juntamente}
            \MAG{com Cristo}, --- pela graça sois salvos,
            %-----!j 92 -i12
            \ver{6}~e, \YEL{juntamente com ele, nos ressuscitou}, e \CYA{nos fez assentar nos
            lugares celestiais} \MAG{em Cristo Jesus};
        }
        \\[1.5\bigskipamount]
        \begin{enumerate}
            \item<2-> Fomos \BRI{colocados na presença de Deus, em Cristo Jesus};
            \item<3-> Vida entronizada: reinar sobre \BRI{circunstâncias} / \BRI{situações};
            \item<4-> Vida de \BRI{oração} / vida de \BRI{louvor} / \BRI{intimidade} com Deus;
            \item<5-> \BRI{Posição} de recebermos revelação do que Deus quer para nós \BRI{em
                seguida}.
        \end{enumerate}
    \end{frame}

%-----------------------------------------------------------------------------------------------
\section{Aplicação}
%-----------------------------------------------------------------------------------------------

    %-------------------------------------------------------------------------------------------
    \subsection{Verdades Experimentadas Pela Fé}
    %-------------------------------------------------------------------------------------------

    \begin{frame}{1Jo 1.9 (ARA)}
        \QUOTE{%
            %-----!j 92 -i12
            \ver{9}~Se confessarmos os nossos pecados, \YEL{ele é fiel e justo para nos perdoar}
            os pecados e nos purificar de toda injustiça.
        }
    \end{frame}

%-----------------------------------------------------------------------------------------------
\section{Os Cinco Imperativos de Rm 6.11--13}
%-----------------------------------------------------------------------------------------------

    %-------------------------------------------------------------------------------------------
    \subsection{Visão Geral}
    %-------------------------------------------------------------------------------------------

    \begin{frame}{Os \YEL{Cinco Imperativos} de \YEL{Rm 6.11--13} (ARA)}
        \QUOTE{%
            %-----!j 92 -i12
            \ver{11}~Assim também vós
            \alt<2->{\YEL{considerai-vos mortos para o pecado, mas vivos para
            Deus}}{considerai-vos mortos para o pecado, mas vivos para Deus}, em Cristo Jesus.
            %-----!j 92 -i12
            \ver{12}~%
            \alt<3->{\RED{Não reine, portanto, o pecado em vosso \underline{\emph{corpo}}
            mortal}}{Não reine, portanto, o pecado em vosso corpo mortal}, de maneira que
            obedeçais às suas paixões;
            %-----!j 92 -i12
            \ver{13}~
            \alt<4->{\ORA{nem ofereçais cada um os \underline{\emph{membros}} do seu corpo ao
            pecado}}{nem ofereçais cada um os membros do seu corpo ao pecado}, como instrumentos
            de iniquidade; mas
            \alt<5->{\GRE{oferecei-\underline{\emph{vos}} a Deus, como ressurretos dentre os
            mortos}}{oferecei-vos a Deus, como ressurretos dentre os mortos}, e
            \alt<6->{\CYA{(oferecei) os vossos \underline{\emph{membros}}, a Deus}}{(oferecei)
            os vossos membros, a Deus}, como instrumentos de justiça.
        }
    \end{frame}

    %-------------------------------------------------------------------------------------------
    \subsection{Primeiro Imperativo -- Rm 6.11}
    %-------------------------------------------------------------------------------------------

    \begin{frame}{\YEL{Primeiro Imperativo -- CRER}: Rm 6.11 (ARA)}
        \QUOTE{%
            %-----!j 92 -i12
            \ver{11}~Assim também vós \YEL{considerai-vos mortos para o pecado, mas vivos para
            Deus}, \MAG{em Cristo Jesus}.
        }
        \\[1.5\bigskipamount]
        \uncover<2->{
            \QUOTE{Oração: Pai, te \BRI{agradeço por estar} morto para o pecado / e vivo para
            ti, \MAG{em Cristo Jesus}.}
        }
    \end{frame}

    %-------------------------------------------------------------------------------------------
    \subsection{Segundo Imperativo -- Rm 6.12}
    %-------------------------------------------------------------------------------------------

    \begin{frame}{\YEL{Segundo Imperativo -- ESCOLHER}: Rm 6.12 (ARA)}
        \QUOTE{%
            %-----!j 92 -i12
            \ver{12}~\YEL{Não reine, portanto, o pecado em vosso corpo mortal}, de maneira que
            obedeçais às suas paixões;
        }
        \\[1.5\bigskipamount]
        \uncover<2->{
            \QUOTE{Oração: Pai, eu \BRI{escolho não mais viver pelo meu ponto de vista}.}
        }
    \end{frame}

    %-------------------------------------------------------------------------------------------
    \subsection{Terceiro Imperativo -- Rm 6.13a}
    %-------------------------------------------------------------------------------------------

    \begin{frame}{\YEL{Terceiro Imperativo -- ESCOLHER}: Rm 6.13a (ARA)}
        \QUOTE{%
            %-----!j 92 -i12
            \ver{13}~\YEL{nem ofereçais cada um os membros do seu corpo ao pecado}, como
            instrumentos de iniquidade;
        }
        \\[1.5\bigskipamount]
        \uncover<2->{
            \QUOTE{Oração: Pai, eu \BRI{escolho não entregar qualquer membro do meu corpo ao meu
            ponto de vista}.}
        }
    \end{frame}

    \begin{frame}{\BRI{Despojar-se do velho homem}: Rm 6.11--13a (ARA)}
        \QUOTE{
            Oração:\normalsize
            \\[\bigskipamount]
            Pai, eu \YEL{creio} que estou morto para o \RED{pecado} e vivo para ti, em Cristo Jesus.
            \\[\medskipamount]
            Eu \YEL{escolho} não mais viver pelo \RED{meu ponto de vista}.
            \\[\medskipamount]
            Eu \YEL{escolho} não entregar qualquer membro do meu corpo ao \RED{meu ponto de vista}.
        }
    \end{frame}

    %-------------------------------------------------------------------------------------------
    \subsection{Quarto Imperativo -- Rm 6.13b}
    %-------------------------------------------------------------------------------------------

    \begin{frame}{\YEL{Quarto Imperativo -- CRER e ESCOLHER}: Rm 6.13b (ARA)}
        \QUOTE{%
            %-----!j 92 -i12
            \ver{13}~mas \YEL{oferecei-vos a Deus, como ressurretos dentre os mortos},
        }
        \\[1.5\bigskipamount]
        \uncover<2->{
            \QUOTE{
                Oração: mas \BRI{eu me ofereço a ti, ó Pai, em Jesus Cristo, como uma pessoa
                ressurreta}.
            }
        }
    \end{frame}

    \begin{frame}{\YEL{Diante do Trono}: Hb 4.16 (ARA)}
        \QUOTE{%
            %-----!j 92 -i12
            \ver{16}~Acheguemo-nos, portanto, confiadamente, \YEL{junto ao trono da graça}, a
            fim de recebermos \GRE{misericórdia} e acharmos \CYA{graça para socorro} em ocasião
            oportuna.
        }
    \end{frame}

    %-------------------------------------------------------------------------------------------
    \subsection{Quinto Imperativo -- Rm 6.13c}
    %-------------------------------------------------------------------------------------------

    \begin{frame}{\YEL{Quinto Imperativo -- ESCOLHER}: Rm 6.13c (ARA)}
        \QUOTE{%
            %-----!j 92 -i12
            \ver{13}~e \YEL{(oferecei) os vossos membros, a Deus, como instrumentos de justiça}.
        }
        \\[1.5\bigskipamount]
        \uncover<2->{
            \QUOTE{
                Oração: e \BRI{ofereço a ti, ó Deus, os meus membros, como instrumentos da tua
                justiça}.
            }
        }
    \end{frame}

    \begin{frame}{\BRI{Vestir-se do novo homem}: Rm 6.13 (ARA)}
        \QUOTE{
            Oração (continuação):\normalsize
            \\[\bigskipamount]
            mas eu me \YEL{ofereço} a ti, ó Pai, em Jesus Cristo, como uma pessoa ressurreta,
            \\[\medskipamount]
            e \YEL{ofereço} a ti, ó Deus, os meus membros, como instrumentos da tua justiça.
        }
    \end{frame}

%-----------------------------------------------------------------------------------------------
\section{Resultado}
%-----------------------------------------------------------------------------------------------

    %-------------------------------------------------------------------------------------------
    \subsection{Consequência da Obediência -- Rm 6.14}
    %-------------------------------------------------------------------------------------------

    \begin{frame}{Rm 6.14 (ARA) -- Resultado da \YEL{Obediência}}
        \QUOTE{%
            %-----!j 92 -i12
            \ver{14}~\alt<2->{\ORA{Então}}{Porque} \YEL{o pecado não terá domínio sobre vós};
            pois não estais \RED{debaixo da lei}, \GRE{e sim da graça}.
        }
    \end{frame}

    %-------------------------------------------------------------------------------------------
    \subsection{Andar por Graça}
    %-------------------------------------------------------------------------------------------

    \begin{frame}{Jo 14.10--12}
        \QUOTE{%
            %-----!j 92 -i12
            \ver{10}~Não crês que eu \BRI{estou no Pai e que o Pai está em mim}? As palavras que
            eu vos digo \YEL{não as digo por mim mesmo}; mas \GRE{o Pai}, que permanece em mim,
            \GRE{faz as suas obras}.
            %-----!j 92 -i12
            \ver{11}~Crede-me que \BRI{estou no Pai, e o Pai, em mim}; crede ao menos \CYA{por
            causa das mesmas obras}.
            %-----!j 92 -i12
            \ver{12}~Em verdade, em verdade vos digo que \YEL{aquele que crê em mim} \CYA{fará
            também as obras que eu faço e outras maiores fará}, porque eu vou para junto do Pai.
        }
    \end{frame}

%-----------------------------------------------------------------------------------------------
\end{document}
%-----------------------------------------------------------------------------------------------

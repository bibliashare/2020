%-----------------------------------------------------------------------------------------------
\documentclass[12pt,aspectratio=169]{beamer}
%-----------------------------------------------------------------------------------------------
\usepackage{pslatex}
%-----------------------------------------------------------------------------------------------
\newcommand{\YA}{%
    \mbox{%
        Y\makebox[0pt][l]{\hspace{-0.178em}\raisebox{-0.00ex}{\scalebox{0.30}{E}}}%
        H\makebox[0pt][l]{\hspace{-0.010em}\raisebox{-0.00ex}{\scalebox{0.30}{O}}}%
        W\makebox[0pt][l]{\hspace{-0.245em}\raisebox{-0.00ex}{\scalebox{0.30}{A}}}%
        H%
    }%
}
%-----------------------------------------------------------------------------------------------
\newcommand{\ver}[1]{%
    \raisebox{0.50ex}{%
        \scalebox{1.1}{%
            \pmb{\textbf{\textcolor{BSpbg}{#1}}}%
        }%
    }%
}
%-----------------------------------------------------------------------------------------------
\newcommand{\QUOTE}[1]{%
    \par\noindent\hspace*{0.05\linewidth}%
    \begin{minipage}{0.9\linewidth}%
        \linespread{1.35}\large{#1}%
    \end{minipage}%
}
%-----------------------------------------------------------------------------------------------
\newcommand{\RED}[1]{{\textcolor{TXred}{#1}}}
\newcommand{\ORA}[1]{{\textcolor{TXred!50!TXyel}{#1}}}
\newcommand{\YEL}[1]{{\textcolor{TXyel}{#1}}}
\newcommand{\GRE}[1]{{\textcolor{TXgre}{#1}}}
\newcommand{\CYA}[1]{{\textcolor{TXcya}{#1}}}
\newcommand{\BLU}[1]{{\textcolor{TXblu}{#1}}}
\newcommand{\MAG}[1]{{\textcolor{TXmag}{#1}}}
\newcommand{\BRI}[1]{{\textcolor{BSpbg}{#1}}}   % Bright
%-----------------------------------------------------------------------------------------------
\usetheme{CambridgeUS}
\usefonttheme{serif}
\usecolortheme{BShare1}
%-----------------------------------------------------------------------------------------------
\title{Reino de Cristo}
\subtitle{Fundamentos}
\author{Bíblia Share}
%\institute{Bíblia Share}
\date[{\tiny\tt 19 de Novembro de 2020}]{{\scriptsize\tt%
    \includegraphics[height=6.0mm]{res/cc/by-nc-nd-88x31.pdf}\\[\smallskipamount]
    19 de Novembro de 2020
}}
%-----------------------------------------------------------------------------------------------
\begin{document}
%-----------------------------------------------------------------------------------------------
\logo{%
    \parbox{158mm}{% There's a 1mm gap on each side of the 160mm x 90mm slide logo line
    \mode<beamer>{%
        \hfill\includegraphics[height=9.0mm]{res/logo/BibliaShare.pdf}%
    }
    \mode<handout>{%
        \hfill\includegraphics[height=9.0mm]{res/logo/BibliaShare.pdf}%
    }
}}
%-----------------------------------------------------------------------------------------------
\begin{frame}
    \titlepage
\end{frame}
%-----------------------------------------------------------------------------------------------
\section{O Primeiro Adão}
%-----------------------------------------------------------------------------------------------

    \begin{frame}
        \par\noindent\hspace*{0.05\linewidth}%
        \begin{minipage}{0.9\linewidth}%
            \linespread{1.35}\large%
            \begin{alertblock}{Tópico}
                O Primeiro Adão
            \end{alertblock}
        \end{minipage}%
    \end{frame}

    %-------------------------------------------------------------------------------------------
    \subsection{Imagem de Deus dominando o mundo material}
    %-------------------------------------------------------------------------------------------

    \begin{frame}{Gn 1.1,26,28 (TB)}
        \QUOTE{%
            %-----!j 92 -i12
            \ver{1}~No princípio, criou Deus o \CYA{céu} e a \YEL{terra}. [...]
            %-----!j 92 -i12
            \ver{26}~Disse também Deus: Façamos o \YEL{homem} \MAG{à nossa imagem}, conforme a
            \MAG{nossa semelhança}; \YEL{domine ele} sobre os peixes do mar, sobre as aves do
            céu, sobre os animais domésticos, \YEL{sobre toda a terra} e sobre todo réptil que
            se arrasta sobre a terra.
            %-----!j 92 -i12
            \ver{28}~Deus os abençoou e lhes disse: Frutificai, multiplicai-vos, \YEL{enchei a
            terra} e \YEL{sujeitai-a}; \YEL{dominai} sobre os peixes do mar, sobre as aves do
            céu e sobre todos os animais que se arrastam sobre a terra.
        }
    \end{frame}

    \begin{frame}{Ez 28.13a,14--15 (ARA), 16b (TB)}
        \QUOTE{%
            %-----!j 92 -i12
            \ver{13}~\YEL{Estavas no Éden}, jardim de Deus; [...]
            %-----!j 92 -i12
            \ver{14}~Tu eras \YEL{querubim} da \YEL{guarda} ungido, e te \YEL{estabeleci};
            permanecias no monte santo de Deus, no brilho das pedras andavas.
            %-----!j 92 -i12
            \ver{15}~\YEL{Perfeito} eras nos teus caminhos, desde o dia em que foste
            \YEL{criado} até que se achou \RED{iniquidade} em ti.
            %-----!j 92 -i12
            \ver{16}~[...] e \RED{pecaste}; portanto te \RED{lancei, profanado}, do monte de
            Deus, e te exterminei, ó \YEL{querubim} cobridor, [...]
        }
    \end{frame}

    \begin{frame}{Jo 8.34,44; 10.10 (TB)}
        \QUOTE{%
            %-----!j 92 -i12
            \ver{8.34}~Replicou-lhes Jesus: Em verdade, em verdade vos digo: \YEL{todo} o que
            comete \RED{pecado} é \RED{escravo do pecado}.
            %-----!j 92 -i12
            \ver{8.44}~Vós sois filhos do \RED{Diabo} e tendes vontade de cumprir os
            \RED{desejos} de vosso pai. Ele era \RED{homicida desde o princípio} e não
            permaneceu na verdade, porque não há nele verdade. Quando ele diz uma mentira, fala
            do que lhe é próprio, porque é \RED{mentiroso} e o pai da mentira.
            %-----!j 92 -i12
            \ver{10.10}~O ladrão não vem senão para \RED{furtar, matar e destruir}; eu vim para
            que elas tenham vida e a tenham em abundância.
        }
    \end{frame}

    %-------------------------------------------------------------------------------------------
    \subsection{Queda, Maldição e Promessas de Restauração}
    %-------------------------------------------------------------------------------------------

    \begin{frame}{Gn 3.17--19 (TB)}
        \QUOTE{%
            %-----!j 92 -i12
            \ver{17}~E a Adão disse: Porque escutaste a voz de tua mulher e comeste da árvore de
            que te ordenei que não comesses, \RED{maldita é a terra por tua causa}; em fadiga
            tirarás dela o sustento todos os \RED{dias da tua vida}.
            %-----!j 92 -i12
            \ver{18}~Ela te produzirá também \RED{espinhos e abrolhos}, e comerás as ervas do
            campo.
            %-----!j 92 -i12
            \ver{19}~No \RED{suor do teu rosto comerás o teu pão}, até que te tornes à terra,
            pois dela foste tomado: porquanto \RED{tu és pó e em pó te hás de tornar}.
        }
    \end{frame}

    \begin{frame}{Gn 3.14 (TB) 15 (ARA)}
        \QUOTE{%
            %-----!j 92 -i12
            \ver{14}~Então, disse Deus Jeová à \RED{serpente}: Porquanto assim o fizeste,
            \RED{maldita és} tu dentre todos os animais domésticos e dentre todos os animais do
            campo; sobre o teu ventre andarás de rastos, o pó comerás todos os \RED{dias da tua
            vida}.
            %-----!j 92 -i12
            \ver{15}~Porei inimizade entre \RED{ti} e a \YEL{mulher}, entre a \RED{tua
            descendência} e \YEL{o seu descendente}.  / \YEL{Este te ferirá a cabeça}, e \RED{tu
            lhe ferirás o calcanhar}.
        }
    \end{frame}

%-----------------------------------------------------------------------------------------------
\section{Período de Consiência}
%-----------------------------------------------------------------------------------------------

    \begin{frame}
        \par\noindent\hspace*{0.05\linewidth}%
        \begin{minipage}{0.9\linewidth}%
            \linespread{1.35}\large%
            \begin{alertblock}{Tópico}
                Período de Consiência
            \end{alertblock}
        \end{minipage}%
    \end{frame}

    %-------------------------------------------------------------------------------------------
    \subsection{Violência e Caos da Humanidade Caída}
    %-------------------------------------------------------------------------------------------

    \begin{frame}{Gn 4.1,2,8 (TB)}
        \QUOTE{%
            %-----!j 92 -i12
            \ver{1}~O homem conheceu a Eva, sua mulher; ela concebeu e, dando à luz a
            \YEL{Caim}\footnote{\uncover<2->{\YEL{Caim}: `uma lança,' `ferreiro,' (Cambridge
            Bible) ou `possuir,' `estabelecer' --- talvez pensando que fosse a prometida
            semente?  (Henry's).}}, disse: Adquiri um homem com o auxílio de Jeová.
            %-----!j 92 -i12
            \ver{2}~Tornou a dar à luz a um filho, a
            \YEL{Abel}\footnote{\uncover<3->{\YEL{Abel}: `vaidade,' `sopro' --- talvez pensando
            ser inútil ter outro filho? (Henry's).}}, seu irmão. Abel foi pastor de ovelhas, mas
            Caim foi lavrador da terra.
            %-----!j 92 -i12
            \ver{8}~Sucedeu, pois, que, estando eles no campo, se levantou Caim contra seu irmão
            Abel \RED{e o matou}.
        }
    \end{frame}

    \begin{frame}{Gn 4.23,24 (TB)}
        \QUOTE{%
            %-----!j 92 -i12
            \ver{23}~Disse Lameque\footnote{Lameque: A quinta geração de Caim} à suas mulheres:
            Ada e Zilá, ouvi a minha voz; Vós, mulheres de Lameque, escutai as minhas palavras:
            pois \RED{matei um homem, porque me feriu}; e um \RED{mancebo, porque me pisou}.
            %-----!j 92 -i12
            \ver{24}~Se por Caim tomar-se-á vingança sete vezes, com certeza, por Lameque o será
            setenta e sete vezes\footnote{Pelo tempo típico entre gerações, no Cap.~5, a palavra
            de Deus é recitada $\approx 500$ anos após o fato, indicando forte tradição oral.}.
        }
    \end{frame}

    \begin{frame}{Gn 5.28,29 (TB)}
        \QUOTE{%
            %-----!j 92 -i12
            \ver{28}~Lameque\footnote{Este é descendente de Sete, não de Caim. Seguindo a
            genealogia do Cap.~5, a frase do v.~29 vem \BRI{$1056$ anos após a criação}, e
            \BRI{$126$ anos após a morte de Adão} --- tradição oral.} viveu cento e oitenta e
            dois anos e gerou um filho,
            %-----!j 92 -i12
            \ver{29}~a quem chamou Noé, dizendo: Este nos dará \YEL{descanso das nossas obras e
            do trabalho} das nossas mãos, \YEL{os quais vêm da terra que Jeová amaldiçoou}.
        }
    \end{frame}

    \begin{frame}{Gn 6.5--8 (TB)}
        \QUOTE{%
            %-----!j 92 -i12
            \ver{5}~Viu Jeová que era \RED{grande a maldade} do homem na terra e que \YEL{toda}
            a imaginação dos \RED{pensamentos} do seu coração era \RED{má} \YEL{continuamente}.
            %-----!j 92 -i12
            \ver{6}~Então, se arrependeu\footnote{\YEL{Nacham}: \BRI{suspirar}, \BRI{respirar
            com força} (Strong's Exhaustive Concordance).} Jeová de ter feito o homem na terra,
            e pesou-lhe em seu coração.
        }
    \end{frame}

    \begin{frame}{Gn 6.5--8 (TB)}
        \QUOTE{%
            %-----!j 92 -i12
            \ver{7}~Disse Jeová: Farei \RED{desaparecer da face da terra o homem que criei},
            desde o homem até o animal, até os répteis e até as aves do céu; porque me arrependo
            de os haver feito.
            %-----!j 92 -i12
            \ver{8}~\GRE{Porém Noé achou graça aos olhos de Jeová}.
        }
    \end{frame}

%-----------------------------------------------------------------------------------------------
\section{Aliança com Noé (Governo Humano)}
%-----------------------------------------------------------------------------------------------

    \begin{frame}
        \par\noindent\hspace*{0.05\linewidth}%
        \begin{minipage}{0.9\linewidth}%
            \linespread{1.35}\large%
            \begin{alertblock}{Tópico}
                Aliança com Noé (Governo Humano)
            \end{alertblock}
        \end{minipage}%
    \end{frame}

    %-------------------------------------------------------------------------------------------
    \subsection{Aliança com Noé}
    %-------------------------------------------------------------------------------------------

    \begin{frame}{Gn 9.1--4 (TB)}
        \QUOTE{%
            %-----!j 92 -i12
            \ver{1}~Abençoou Deus a Noé e a seus filhos e lhes disse: \YEL{Frutificai,
            multiplicai-vos e enchei a terra}.
            %-----!j 92 -i12
            \ver{2}~Terá \ORA{medo e pavor de vós todo o animal da terra e toda a ave do céu};
            nas vossas mãos serão eles entregues juntamente com tudo o que se move sobre a terra
            e com todos os peixes do mar.
            %-----!j 92 -i12
            \ver{3}~Tudo o que se move e vive vos servirá de mantimento; \YEL{como a erva verde,
            tudo vos tenho dado a vós}.
            %-----!j 92 -i12
            \ver{4}~A carne, porém, com sua vida, isto é, com seu \RED{sangue, não comereis}.
        }
    \end{frame}

    \begin{frame}{Gn 9.5,6 (TB)}
        \QUOTE{%
            %-----!j 92 -i12
            \ver{5}~\YEL{Certamente, requererei o vosso sangue}, o sangue das vossas vidas;
            \ORA{da mão de todo o animal, o requererei}; e, \RED{da mão do homem}, sim, da mão
            do irmão de cada um, requererei a vida do homem.
            %-----!j 92 -i12
            \ver{6}~\YEL{Se alguém derramar o sangue do homem}, \GRE{pelo homem será derramado o
            seu sangue}; porque \CYA{o homem foi feito à imagem de Deus}.
        }
    \end{frame}

    \begin{frame}{Gn 9.12,13 (TB)}
        \QUOTE{%
            %-----!j 92 -i12
            \ver{12}~Disse Deus: Este é o \YEL{sinal da aliança} que faço entre mim e vós e todo
            o animal vivente que está convosco, para \YEL{perpétuas gerações}:
            %-----!j 92 -i12
            \ver{13}~\YEL{o meu arco} tenho posto nas nuvens, e será ele por sinal de uma
            \YEL{aliança entre mim e a terra}.
        }
    \end{frame}

    \begin{frame}{Ap 4.2,3 (TB)}
        \QUOTE{%
            %-----!j 92 -i12
            \ver{2}~Imediatamente, fui arrebatado pelo Espírito. Eis havia um \YEL{trono posto
            no céu}, e, sobre o trono, um sentado;
            %-----!j 92 -i12
            \ver{3}~e aquele que estava sentado era, pelo que parecia, semelhante a uma pedra de
            jaspe e de sardônio; e \YEL{havia ao redor do trono um arco-íris} semelhante, pelo
            que parecia, à \GRE{esmeralda}.
        }
    \end{frame}

    {\usebackgroundtemplate{\parbox{\paperwidth}{
        \vspace*{1sp}\centering\includegraphics[height=\paperheight]{{fig/emerald.jpg}}
    }}\frame[plain]{%
        \vspace*{72mm}\color{white}\scriptsize\bf{Rough emerald crystals from Panjshir
        Valley Afghanistan by Paweł Maliszczak}
    }\usebackgroundtemplate{\mbox{~}}}

    %-------------------------------------------------------------------------------------------
    \subsection{Torre de Babel e Surgimento das Nações}
    %-------------------------------------------------------------------------------------------

    \begin{frame}{Gn 11.4--9 (TB)}
        \QUOTE{%
            %-----!j 92 -i12
            \ver{4}~E disseram: Vinde, edifiquemos para nós uma cidade e uma torre cujo cume
            chegue até o céu e façamo-nos um nome; para que \RED{não sejamos espalhados} sobre a
            face de toda a terra.
            %-----!j 92 -i12
            \ver{5}~Porém, desceu Jeová para ver a cidade e a torre que os filhos dos homens
            edificavam.
            %-----!j 92 -i12
            \ver{6}~Disse Jeová: Eis que o povo é um só, e todos eles têm uma só linguagem.
            \ORA{Isso é o que começam} a fazer: agora, nada lhes será vedado de quanto intentam
            fazer.
        }
    \end{frame}

    \begin{frame}{Gn 11.4--9 (TB)}
        \QUOTE{%
            %-----!j 92 -i12
            \ver{7}~Vinde, desçamos e confundamos ali a sua linguagem, para que não entendam a
            linguagem um do outro.
            %-----!j 92 -i12
            \ver{8}~Assim Jeová os \YEL{espalhou dali} sobre a face de toda a terra; e cessaram
            de edificar a cidade.
            %-----!j 92 -i12
            \ver{9}~Por isso, se chamou o seu nome \YEL{Babel}, porquanto ali \YEL{confundiu}
            Jeová a linguagem de toda a terra; e dali os \YEL{espalhou} sobre a face de toda a
            terra.
        }
    \end{frame}

%-----------------------------------------------------------------------------------------------
\section{Alianças com Abrão (Patriarca de Israel)}
%-----------------------------------------------------------------------------------------------

    \begin{frame}
        \par\noindent\hspace*{0.05\linewidth}%
        \begin{minipage}{0.9\linewidth}%
            \linespread{1.35}\large%
            \begin{alertblock}{Tópico}
                Alianças com Abrão (Patriarca de Israel)
            \end{alertblock}
        \end{minipage}%
    \end{frame}

    %-------------------------------------------------------------------------------------------
    \subsection{Chamado e Promessas \BRI{Incondicionais}}
    %-------------------------------------------------------------------------------------------

    \begin{frame}{Gn 12.1--3 (TB)}
        \QUOTE{%
            %-----!j 92 -i12
            \ver{1}~Ora, disse Jeová a Abrão: Sai-te da tua terra, da tua parentela e da casa de
            teu pai para a \ORA{terra} que te mostrarei;
            %-----!j 92 -i12
            \ver{2}~farei de ti uma \YEL{grande nação}, e te \GRE{abençoarei}, e
            \CYA{engrandecerei} o teu nome.  \GRE{Sê tu uma bênção}.
            %-----!j 92 -i12
            \ver{3}~\BLU{Abençoarei os que te abençoarem} e \RED{amaldiçoarei aquele que te
            amaldiçoar}; por meio de ti, serão \MAG{benditas todas as famílias da terra}.
        }
    \end{frame}

    \begin{frame}{Mt 1.1; Gl 3.1,8 (TB)}
        \QUOTE{%
            %-----!j 92 -i12
            \ver{1}~Livro da geração de \MAG{Jesus Cristo}, filho de Davi, \YEL{filho de
            Abraão}.
            %-----!j 92 -i12
            \ver{1}~Ó insensatos gálatas! Quem vos fascinou a vós, ante cujos olhos foi
            representado \MAG{Jesus Cristo} como crucificado?
            %-----!j 92 -i12
            \ver{8}~A \YEL{Escritura}, \YEL{prevendo} que Deus justificaria os gentios pela fé,
            \YEL{de antemão, anunciou} as boas-novas a Abraão: \MAG{Em ti serão bem-aventuradas
            todas as nações}.
        }
    \end{frame}

%-----------------------------------------------------------------------------------------------
\end{document}
%-----------------------------------------------------------------------------------------------

%-----------------------------------------------------------------------------------------------
\documentclass[12pt,aspectratio=169]{beamer}
%-----------------------------------------------------------------------------------------------
\usepackage{pslatex}
%-----------------------------------------------------------------------------------------------
\newcommand{\YA}{%
    \mbox{%
        Y\makebox[0pt][l]{\hspace{-0.178em}\raisebox{-0.00ex}{\scalebox{0.30}{E}}}%
        H\makebox[0pt][l]{\hspace{-0.010em}\raisebox{-0.00ex}{\scalebox{0.30}{O}}}%
        W\makebox[0pt][l]{\hspace{-0.245em}\raisebox{-0.00ex}{\scalebox{0.30}{A}}}%
        H%
    }%
}
%-----------------------------------------------------------------------------------------------
\newcommand{\ver}[1]{%
    \raisebox{0.50ex}{%
        \scalebox{1.1}{%
            \pmb{\textbf{\textcolor{BSpbg}{#1}}}%
        }%
    }%
}
%-----------------------------------------------------------------------------------------------
\newcommand{\QUOTE}[1]{%
    \par\noindent\hspace*{0.05\linewidth}%
    \begin{minipage}{0.9\linewidth}%
        \linespread{1.35}\large{#1}%
    \end{minipage}%
}
%-----------------------------------------------------------------------------------------------
\newcommand{\parallelsum}{\mathbin{\!/\mkern-5mu/\!}} % https://tex.stackexchange.com/a/37917
%-----------------------------------------------------------------------------------------------
\newcommand{\RED}[1]{{\textcolor{TXred}{#1}}}
\newcommand{\ORA}[1]{{\textcolor{TXred!50!TXyel}{#1}}}
\newcommand{\YEL}[1]{{\textcolor{TXyel}{#1}}}
\newcommand{\GRE}[1]{{\textcolor{TXgre}{#1}}}
\newcommand{\CYA}[1]{{\textcolor{TXcya}{#1}}}
\newcommand{\BLU}[1]{{\textcolor{TXblu}{#1}}}
\newcommand{\MAG}[1]{{\textcolor{TXmag}{#1}}}
\newcommand{\BRI}[1]{{\textcolor{BSpbg}{#1}}}   % Bright
%-----------------------------------------------------------------------------------------------
\usetheme{CambridgeUS}
\usefonttheme{serif}
\usecolortheme{BShare1}
%-----------------------------------------------------------------------------------------------
\title[Reino de Cristo -- Princípios]{Reino de Cristo}
\subtitle{Princípios}
\author{Bíblia Share}
%\institute{Bíblia Share}
\date[{\tiny\tt 16 de Dezembro de 2020}]{{\scriptsize\tt%
    \includegraphics[height=6.0mm]{res/cc/by-nc-nd-88x31.pdf}\\[\smallskipamount]
    16 de Dezembro de 2020\\
    Christian Naaktgeboren
}}
%-----------------------------------------------------------------------------------------------
\begin{document}
%-----------------------------------------------------------------------------------------------
\logo{%
    \parbox{158mm}{% There's a 1mm gap on each side of the 160mm x 90mm slide logo line
    \mode<beamer>{%
        \hfill\includegraphics[height=9.0mm]{res/logo/BibliaShare.pdf}%
    }
    \mode<handout>{%
        \hfill\includegraphics[height=9.0mm]{res/logo/BibliaShare.pdf}%
    }
}}
%-----------------------------------------------------------------------------------------------
\begin{frame}
    \titlepage
\end{frame}
%-----------------------------------------------------------------------------------------------
\section{De Abraão às Doze Tribos de Israel}
%-----------------------------------------------------------------------------------------------

    \begin{frame}
        \par\noindent\hspace*{0.05\linewidth}%
        \begin{minipage}{0.9\linewidth}%
            \large%
            \begin{alertblock}{Tópico}
                De Abraão às Doze Tribos de Israel
            \end{alertblock}
        \end{minipage}%
    \end{frame}

    %-------------------------------------------------------------------------------------------
    \subsection{Isaque}
    %-------------------------------------------------------------------------------------------

    \begin{frame}[allowframebreaks]{Cumprimento da Promessa de \YEL{Isaque} --}
        \QUOTE{%
            %-----!j 92 -i12
            \ver{Gn 21.1}~Jeová \BRI{visitou} a Sara, \MAG{como dissera}, e \BRI{fez}-lhe
            \MAG{como tinha falado}.
            %-----!j 92 -i12
            \ver{2}~Ela concebeu e deu à luz a Abraão, na sua velhice, um filho, \MAG{ao tempo
            determinado}, de que \BLU{Deus} lhe tinha \BLU{falado}.
            %-----!j 92 -i12
            \ver{3}~Ao filho que lhe nascera, que Sara lhe dera à luz, Abraão chamou
            \YEL{Isaque}.
            %-----!j 92 -i12
            \ver{4}~E ele \YEL{circuncidou} a seu filho Isaque, quando tinha \YEL{oito dias},
            \YEL{conforme Deus lhe ordenara}.
        }

        \pagebreak

        \QUOTE{%
            %-----!j 92 -i12
            \ver{5}~Abraão tinha \GRE{cem anos}, quando Isaque, seu filho, lhe nasceu.
            %-----!j 92 -i12
            \ver{6}~Disse Sara: Deus preparou \GRE{riso} para mim; todo aquele que o ouvir se
            rirá por minha causa.
            %-----!j 92 -i12
            \ver{7}~E continuou: \BLU{Quem} teria dito a Abraão que Sara havia de amamentar
            filhos?  Pois, na sua velhice, lhe dei um filho.
        }
    \end{frame}

    \begin{frame}[allowframebreaks]{Descendência de Abraão é por \YEL{Isaque} --}
        \QUOTE{%
            %-----!j 92 -i12
            \ver{Gn 21.8}~Isaque cresceu e foi \BRI{desmamado}. Nesse dia em que o menino foi
            desmamado, deu Abraão um grande banquete.
            %-----!j 92 -i12
            \ver{9}~Vendo Sara que o filho de Agar, a egípcia, o qual ela dera à luz a Abraão,
            \RED{caçoava} de Isaque,
            %-----!j 92 -i12
            \ver{10}~disse a Abraão: Rejeita essa escrava e seu filho; \ORA{porque o filho dessa
            escrava não será herdeiro com Isaque, meu filho}.
        }

        \pagebreak

        \QUOTE{%
            %-----!j 92 -i12
            \ver{11}~Pareceu isso mui penoso aos olhos de Abraão, por causa de seu filho.
            %-----!j 92 -i12
            \ver{12}~Disse, porém, Deus a Abraão: Não te pareça isso mal por causa do moço e por
            causa da tua serva; \YEL{atende a Sara} em tudo o que ela te disser; porque \YEL{por
            Isaque será chamada a tua descendência}.
            %-----!j 92 -i12
            \ver{13}~Mas também do filho da serva \ORA{farei uma grande nação, por ser ele teu
            descendente}.
        }
    \end{frame}

    \begin{frame}{\YEL{Isaque} e \MAG{Cristo}}
        \begin{itemize}
            \item \MAG{Deus} prova Abraão, pedindo-lhe seu \YEL{único filho Isaque} em
                holocausto;
            \item O \MAG{Anjo de \YA} brada do céu quando Abraão tiha o cutelo na mão para
                imolar o filho: ``agora sei que temes a \BLU{Deus}, porquanto não \BLU{me}
                negaste o filho, o teu \YEL{único filho}''.
            \item \MAG{\YA} provê o carneiro para o holocausto, o qual é oferecido no lugar de
                Isaque.
        \end{itemize}

        \vspace{1.5\bigskipamount}

        \QUOTE{
            %-----!j 92 -i12
            \ver{Gn 22.14}~Chamou Abraão àquele lugar \MAG{Jeová-Jiré}---\MAG{\YA Proverá}
        }
    \end{frame}

    \begin{frame}[allowframebreaks]{Descendência de \YEL{Isaque} --}
        \QUOTE{%
            %-----!j 92 -i12
            \ver{Gn 25.19}~Estas são as gerações de Isaque, filho de Abraão: \YEL{Abraão} gerou
            a \YEL{Isaque};
            %-----!j 92 -i12
            \ver{20}~e Isaque tinha quarenta anos, quando recebeu por mulher a Rebeca, filha de
            Betuel, o siro de Padã-Arã, irmão de Labão, o siro.
            %-----!j 92 -i12
            \ver{21}~Isaque orou instantemente a Jeová por sua mulher, porque ela era estéril;
            atendeu Jeová às orações de Isaque, e Rebeca, mulher de Isaque, concebeu.
            %-----!j 92 -i12
            \ver{22}~Os filhos lutavam no ventre dela; e ela disse: Se é assim, por que vivo eu?
            E foi consultar a Jeová.
        }

        \pagebreak

        \QUOTE{%
            %-----!j 92 -i12
            \ver{23}~Respondeu-lhe Jeová: \ORA{Duas nações} há no teu ventre, e dois povos se
            \ORA{dividirão} das tuas entranhas: \ORA{um povo será mais forte que o outro}, e
            \ORA{o mais velho servirá ao mais moço}.
            %-----!j 92 -i12
            \ver{24}~Cumpridos que foram os dias para ela dar à luz, eis que gêmeos estavam no
            seu ventre.
            %-----!j 92 -i12
            \ver{25}~Saiu o primeiro, ruivo, todo ele como um vestido de pelo; e chamaram-lhe
            \ORA{Esaú}.
            %-----!j 92 -i12
            \ver{26}~Depois, saiu seu irmão e agarrava com a mão o calcanhar de Esaú; pelo que
            foi chamado \YEL{Jacó}. Isaque tinha sessenta anos, quando Rebeca deu à luz.
        }

        \pagebreak

        \QUOTE{%
            %-----!j 92 -i12
            \ver{Rm 9.10}~E não somente isso, mas também Rebeca, que havia concebido de um, de
            Isaque, nosso pai
            %-----!j 92 -i12
            \ver{11}~(porque não tendo os filhos gêmeos \YEL{ainda nascido}, nem tendo eles
            \YEL{feito bem ou mal} algum, para que o propósito de Deus, segundo a \ORA{eleição},
            ficasse firme, não por causa das obras, mas daquele que chama),
            %-----!j 92 -i12
            \ver{12}~foi dito a ela: \ORA{O mais velho servirá ao mais moço}.
            %-----!j 92 -i12
            \ver{13}~Como está escrito: \ORA{Eu amei a Jacó, porém aborreci a Esaú}.
            %-----!j 92 -i12
            \ver{14}~Que diremos, pois? \GRE{Há, porventura, em Deus injustiça? De modo nenhum!}
        }

        \pagebreak

        \QUOTE{%
            %-----!j 92 -i12
            \ver{1Tm 2.1}~Exorto, pois, antes de tudo, que se façam súplicas, orações,
            intercessões, ações de graças por \GRE{todos os homens},
            %-----!j 92 -i12
            \ver{3}~Isso é \YEL{bom e agradável diante de Deus}, nosso Salvador,
            %-----!j 92 -i12
            \ver{4}~que \GRE{deseja que todos os homens sejam salvos, e que cheguem ao pleno
            conhecimento da verdade}.
            %-----!j 92 -i12
            \ver{5}~Pois só há um Deus e só há um mediador entre Deus e os homens, Cristo Jesus,
            homem,
            %-----!j 92 -i12
            \ver{6}~que se deu a si mesmo em \GRE{resgate por todos}...
        }
    \end{frame}

%-----------------------------------------------------------------------------------------------
\section{Sumário}
%-----------------------------------------------------------------------------------------------

    \begin{frame}
        \par\noindent\hspace*{0.05\linewidth}%
        \begin{minipage}{0.9\linewidth}%
            \large%
            \begin{alertblock}{Sumário --- De Abraão às Doze Tribos de Israel:}
                \normalsize
                \begin{itemize}
					\item<1-> ...
                \end{itemize}
            \end{alertblock}
        \end{minipage}%
    \end{frame}

    %-------------------------------------------------------------------------------------------
    \subsection{Aplicação}
    %-------------------------------------------------------------------------------------------

    \begin{frame}{\YEL{Reino de Cristo} -- Princípios:}
        \begin{itemize}
            \item<1-> Baseia-se no \YEL{acumulado} de \GRE{alianças} e suas \ORA{promessas};
            \item<1-> ...
        \end{itemize}
    \end{frame}

%-----------------------------------------------------------------------------------------------
\end{document}
%-----------------------------------------------------------------------------------------------

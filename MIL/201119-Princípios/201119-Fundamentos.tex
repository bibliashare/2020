%-----------------------------------------------------------------------------------------------
\documentclass[12pt,aspectratio=169]{beamer}
%-----------------------------------------------------------------------------------------------
\usepackage{pslatex}
%-----------------------------------------------------------------------------------------------
\newcommand{\YA}{%
    \mbox{%
        Y\makebox[0pt][l]{\hspace{-0.178em}\raisebox{-0.00ex}{\scalebox{0.30}{E}}}%
        H\makebox[0pt][l]{\hspace{-0.010em}\raisebox{-0.00ex}{\scalebox{0.30}{O}}}%
        W\makebox[0pt][l]{\hspace{-0.245em}\raisebox{-0.00ex}{\scalebox{0.30}{A}}}%
        H%
    }%
}
%-----------------------------------------------------------------------------------------------
\newcommand{\ver}[1]{%
    \raisebox{0.50ex}{%
        \scalebox{1.1}{%
            \pmb{\textbf{\textcolor{BSpbg}{#1}}}%
        }%
    }%
}
%-----------------------------------------------------------------------------------------------
\newcommand{\QUOTE}[1]{%
    \par\noindent\hspace*{0.05\linewidth}%
    \begin{minipage}{0.9\linewidth}%
        \linespread{1.35}\large{#1}%
    \end{minipage}%
}
%-----------------------------------------------------------------------------------------------
\newcommand{\parallelsum}{\mathbin{\!/\mkern-5mu/\!}} % https://tex.stackexchange.com/a/37917
%-----------------------------------------------------------------------------------------------
\newcommand{\RED}[1]{{\textcolor{TXred}{#1}}}
\newcommand{\ORA}[1]{{\textcolor{TXred!50!TXyel}{#1}}}
\newcommand{\YEL}[1]{{\textcolor{TXyel}{#1}}}
\newcommand{\GRE}[1]{{\textcolor{TXgre}{#1}}}
\newcommand{\CYA}[1]{{\textcolor{TXcya}{#1}}}
\newcommand{\BLU}[1]{{\textcolor{TXblu}{#1}}}
\newcommand{\MAG}[1]{{\textcolor{TXmag}{#1}}}
\newcommand{\BRI}[1]{{\textcolor{BSpbg}{#1}}}   % Bright
%-----------------------------------------------------------------------------------------------
\usetheme{CambridgeUS}
\usefonttheme{serif}
\usecolortheme{BShare1}
%-----------------------------------------------------------------------------------------------
\title[Reino de Cristo -- Princípios]{Reino de Cristo}
\subtitle{Princípios}
\author{Bíblia Share}
%\institute{Bíblia Share}
\date[{\tiny\tt 19 de Novembro de 2020}]{{\scriptsize\tt%
    \includegraphics[height=6.0mm]{res/cc/by-nc-nd-88x31.pdf}\\[\smallskipamount]
    19 de Novembro de 2020\\
    Christian Naaktgeboren
}}
%-----------------------------------------------------------------------------------------------
\begin{document}
%-----------------------------------------------------------------------------------------------
\logo{%
    \parbox{158mm}{% There's a 1mm gap on each side of the 160mm x 90mm slide logo line
    \mode<beamer>{%
        \hfill\includegraphics[height=9.0mm]{res/logo/BibliaShare.pdf}%
    }
    \mode<handout>{%
        \hfill\includegraphics[height=9.0mm]{res/logo/BibliaShare.pdf}%
    }
}}
%-----------------------------------------------------------------------------------------------
\begin{frame}
    \titlepage
\end{frame}
%-----------------------------------------------------------------------------------------------
\section{O Primeiro Adão}
%-----------------------------------------------------------------------------------------------

    \begin{frame}
        \par\noindent\hspace*{0.05\linewidth}%
        \begin{minipage}{0.9\linewidth}%
            \large%
            \begin{alertblock}{Tópico}
                O Primeiro Adão
            \end{alertblock}
        \end{minipage}%
    \end{frame}

    %-------------------------------------------------------------------------------------------
    \subsection{Imagem de Deus dominando o mundo material}
    %-------------------------------------------------------------------------------------------

    \begin{frame}{Gn 1.1,26,28 (TB)}
        \QUOTE{%
            %-----!j 92 -i12
            \ver{1}~No princípio, criou Deus o \CYA{céu} e a \YEL{terra}. [...]
            %-----!j 92 -i12
            \ver{26}~Disse também Deus: Façamos o \YEL{homem} \MAG{à nossa imagem}, conforme a
            \MAG{nossa semelhança}; \YEL{domine ele} sobre os peixes do mar, sobre as aves do
            céu, sobre os animais domésticos, \YEL{sobre toda a terra} e sobre todo réptil que
            se arrasta sobre a terra.
            %-----!j 92 -i12
            \ver{28}~Deus os abençoou e lhes disse: Frutificai, multiplicai-vos, \YEL{enchei a
            terra} e \YEL{sujeitai-a}; \YEL{dominai} sobre os peixes do mar, sobre as aves do
            céu e sobre todos os animais que se arrastam sobre a terra.
        }
    \end{frame}

    \begin{frame}{Gn 2.15,19 (TB)}
        \QUOTE{%
            %-----!j 92 -i12
            \ver{15}~Tomou, pois, o \MAG{Senhor Deus} ao \GRE{homem} e o colocou no jardim do
            Éden \YEL{para o cultivar e o guardar}.
            %-----!j 92 -i12
            \ver{19}~Havendo, pois, o \MAG{Senhor Deus} formado da terra todos os animais do
            campo e todas as aves dos céus, trouxe-os ao \GRE{homem}, para ver como \YEL{este
            lhes chamaria}; e o nome que o homem desse a todos os seres viventes, \YEL{esse
            seria o nome deles}.
        }
    \end{frame}

    \begin{frame}{Ez 28.13a,14--15 (ARA), 16b (TB)}
        \QUOTE{%
            %-----!j 92 -i12
            \ver{13}~\YEL{Estavas no Éden}, jardim de Deus; [...]
            %-----!j 92 -i12
            \ver{14}~Tu eras \YEL{querubim} da \YEL{guarda} ungido, e te \YEL{estabeleci};
            permanecias no monte santo de Deus, no brilho das pedras andavas.
            %-----!j 92 -i12
            \ver{15}~\YEL{Perfeito} eras nos teus caminhos, desde o dia em que foste
            \YEL{criado} até que se achou \RED{iniquidade} em ti.
            %-----!j 92 -i12
            \ver{16}~[...] e \RED{pecaste}; portanto te \RED{lancei, profanado}, do monte de
            Deus, e te exterminei, ó \YEL{querubim} cobridor, [...]
        }
    \end{frame}

    \begin{frame}{Jo 8.34,44; 10.10 (TB)}
        \QUOTE{%
            %-----!j 92 -i12
            \ver{8.34}~Replicou-lhes Jesus: Em verdade, em verdade vos digo: \YEL{todo} o que
            comete \RED{pecado} é \RED{escravo do pecado}.
            %-----!j 92 -i12
            \ver{8.44}~Vós sois filhos do \RED{Diabo} e tendes vontade de cumprir os
            \RED{desejos} de vosso pai. Ele era \RED{homicida desde o princípio} e não
            permaneceu na verdade, porque não há nele verdade. Quando ele diz uma mentira, fala
            do que lhe é próprio, porque é \RED{mentiroso} e o pai da mentira.
            %-----!j 92 -i12
            \ver{10.10}~O ladrão não vem senão para \RED{furtar, matar e destruir}; eu vim para
            que elas tenham vida e a tenham em abundância.
        }
    \end{frame}

    %-------------------------------------------------------------------------------------------
    \subsection{Queda, Maldição e Promessas de Restauração}
    %-------------------------------------------------------------------------------------------

    \begin{frame}{Gn 2.17; 3.6,7 (TB)}
        \QUOTE{%
            %-----!j 92 -i12
            \ver{2.17}~mas da árvore do \YEL{conhecimento} do \YEL{bem} e do \YEL{mal} não
            comerás; porque, no dia em que dela comeres, certamente morrerás.
            %-----!j 92 -i12
            \ver{3.6}~Vendo a mulher que a árvore era boa para se comer, agradável aos olhos e
            árvore \RED{desejável para dar entendimento}, tomou-lhe do fruto e \RED{comeu} e deu
            também ao marido, e ele \RED{comeu}.
            %-----!j 92 -i12
            \ver{3.7}~Abriram-se, então, os olhos de ambos; e, \YEL{percebendo} que estavam nus,
            \YEL{coseram} folhas de figueira \YEL{e fizeram} cintas para si.
        }
    \end{frame}

    \begin{frame}{Gn 3.17--19 (TB)}
        \QUOTE{%
            %-----!j 92 -i12
            \ver{17}~E a Adão disse: Porque escutaste a voz de tua mulher e comeste da árvore de
            que te ordenei que não comesses, \RED{maldita é a terra por tua causa}; em fadiga
            tirarás dela o sustento todos os \RED{dias da tua vida}.
            %-----!j 92 -i12
            \ver{18}~Ela te produzirá também \RED{espinhos e abrolhos}, e comerás as ervas do
            campo.
            %-----!j 92 -i12
            \ver{19}~No \RED{suor do teu rosto comerás o teu pão}, até que te tornes à terra,
            pois dela foste tomado: porquanto \RED{tu és pó e em pó te hás de tornar}.
        }
    \end{frame}

    \begin{frame}{Gn 3.14 (TB) 15 (ARA)}
        \QUOTE{%
            %-----!j 92 -i12
            \ver{14}~Então, disse Deus Jeová à \RED{serpente}: Porquanto assim o fizeste,
            \RED{maldita és} tu dentre todos os animais domésticos e dentre todos os animais do
            campo; sobre o teu ventre andarás de rastos, o pó comerás todos os \RED{dias da tua
            vida}.
            %-----!j 92 -i12
            \ver{15}~Porei inimizade entre \RED{ti} e a \YEL{mulher}, entre a \RED{tua
            descendência} e \YEL{o seu descendente}.  / \YEL{Este te ferirá a cabeça}, e \RED{tu
            lhe ferirás o calcanhar}.
        }
    \end{frame}

    %-------------------------------------------------------------------------------------------
    \subsection{Sumário}
    %-------------------------------------------------------------------------------------------

    \begin{frame}
        \par\noindent\hspace*{0.05\linewidth}%
        \begin{minipage}{0.9\linewidth}%
            \large%
            \begin{alertblock}{Sumário --- O Primeiro Adão (1 de 2)}
                \normalsize
                \begin{itemize}
                    \item<1-> \ORA{Primeiro} homem \ORA{Adão}, feito à \ORA{imagem de Deus},
                        \ORA{dominando} a \ORA{terra};
                    \item<1-> Inicialmente \ORA{sozinho}, no \ORA{jardim};
                    \item<1-> Depois coletivamente (\ORA{corpo}), por \ORA{toda a terra};
                    \item<1-> \ORA{Deus} propunha, o \ORA{homem} fazia, e o fazia com
                        \ORA{liberdade de fato}\footnote{\uncover<1->{Conf.~``o nome que \YEL{o
                        homem desse} (aos) seres ... \YEL{esse seria o nome deles}.'' -- Gn
                        2.19.}}.
                    \item<1-> O \CYA{último Adão}, Cristo Jesus encarnado, é a \CYA{expressão
                        exata de Deus}\footnote{\uncover<1->{Conf.~Hb 1.3; ``\GRE{imagem do Deus
                        invisível}'' -- Cl 1.15; ``\GRE{eu e o Pai somos um}'' -- Jo 10.30}};
                    \item<1-> Inicialmente \CYA{sozinho}, grão de \CYA{trigo}, caído na terra;
                    \item<1-> Depois coletivamente (\CYA{corpo}), no erguer da cruz, \CYA{atraiu
                        a todos};
                    \item<1-> \CYA{Não age de si mesmo}\footnote{\uncover<1->{Conf.~Jo 5.30,31;
                        ``\GRE{nada faço por mim mesmo; mas como o Pai me ensinou}'' -- 8.28}},
                        e dá \CYA{vida abundante}.
                \end{itemize}
            \end{alertblock}
        \end{minipage}%
    \end{frame}

    \begin{frame}
        \par\noindent\hspace*{0.05\linewidth}%
        \begin{minipage}{0.9\linewidth}%
            \large%
            \begin{alertblock}{Sumário --- O Primeiro Adão (2 de 2)}
                \normalsize
                \begin{itemize}
                    \item<1-> \RED{Maldição} sobre \RED{toda a terra} \ORA{por causa} de
                        Adão\footnote{Mostrando realmente sua então posição de dominador.};
                    \item<1-> \RED{Sustento difícil}: Suor do rosto; espinhos e abrolhos;
                    \item<1-> Duração da vida do homem (e animais) agora \RED{contada em dias};
                    \item<1-> Promessa da \RED{morte} (retorno ao pó);
                    \item<1-> \RED{Inimizade} entre a \RED{descendência da
                        serpente}\footnote{Conf.~Jo 8.44: ``\GRE{Vós sois do diabo, que é vosso
                        pai, e quereis satisfazer-lhe os desejos.}''} e \MAG{o descendente da
                        mulher};
                    \item<1-> Promessa d\MAG{o descendente da mulher} ferir a cabeça da
                        serpente;
                    \item<1-> Confirmação que o homem adquiriu o \RED{conhecimento do (que é)
                        bom/mau}\footnote{Conf.~Gn 3.22: ``\GRE{Eis que o homem se tornou como
                        um de nós, conhecedor do bem e do mal;}''};
                    \item<1-> Homem \RED{lançado fora} do jardim, \RED{sem acesso} à árvore da
                        vida.
                \end{itemize}
            \end{alertblock}
        \end{minipage}%
    \end{frame}

%-----------------------------------------------------------------------------------------------
\section{Período de Consiência}
%-----------------------------------------------------------------------------------------------

    \begin{frame}
        \par\noindent\hspace*{0.05\linewidth}%
        \begin{minipage}{0.9\linewidth}%
            \large%
            \begin{alertblock}{Tópico}
                Período de Consiência
            \end{alertblock}
        \end{minipage}%
    \end{frame}

    %-------------------------------------------------------------------------------------------
    \subsection{Violência e Caos da Humanidade Caída}
    %-------------------------------------------------------------------------------------------

    \begin{frame}{Gn 4.1,2,8 (TB)}
        \QUOTE{%
            %-----!j 92 -i12
            \ver{1}~O homem conheceu a Eva, sua mulher; ela concebeu e, dando à luz a
            \YEL{Caim}\footnote{\uncover<1->{\YEL{Caim}: `uma lança,' `ferreiro,' (Cambridge
            Bible) ou `possuir,' `estabelecer' --- talvez pensando que fosse a prometida
            semente?  (Henry's).}}, disse: Adquiri um homem com o auxílio de Jeová.
            %-----!j 92 -i12
            \ver{2}~Tornou a dar à luz a um filho, a
            \YEL{Abel}\footnote{\uncover<1->{\YEL{Abel}: `vaidade,' `sopro' --- talvez pensando
            ser inútil ter outro filho? (Henry's).}}, seu irmão. Abel foi pastor de ovelhas, mas
            Caim foi lavrador da terra.
            %-----!j 92 -i12
            \ver{8}~Sucedeu, pois, que, estando eles no campo, se levantou Caim contra seu irmão
            Abel \RED{e o matou}.
        }
    \end{frame}

    \begin{frame}{Gn 4.23,24 (TB)}
        \QUOTE{%
            %-----!j 92 -i12
            \ver{23}~Disse Lameque\footnote{Lameque: A quinta geração de Caim} à suas mulheres:
            Ada e Zilá, ouvi a minha voz; Vós, mulheres de Lameque, escutai as minhas palavras:
            pois \RED{matei um homem, porque me feriu}; e um \RED{mancebo, porque me pisou}.
            %-----!j 92 -i12
            \ver{24}~Se por Caim tomar-se-á vingança sete vezes, com certeza, por Lameque o será
            setenta e sete vezes\footnote{Pelo tempo típico entre gerações, no Cap.~5, a palavra
            de Deus é recitada $\approx 500$ anos após o fato, indicando forte tradição oral.}.
        }
    \end{frame}

    \begin{frame}{Gn 5.28,29 (TB)}
        \QUOTE{%
            %-----!j 92 -i12
            \ver{28}~Lameque\footnote{Este é descendente de Sete, não de Caim. Seguindo a
            genealogia do Cap.~5, a frase do v.~29 vem \BRI{$1056$ anos após a criação}, e
            \BRI{$126$ anos após a morte de Adão} --- tradição oral.} viveu cento e oitenta e
            dois anos e gerou um filho,
            %-----!j 92 -i12
            \ver{29}~a quem chamou Noé, dizendo: Este nos dará \YEL{descanso das nossas obras e
            do trabalho} das nossas mãos, \YEL{os quais vêm da terra que Jeová amaldiçoou}.
        }
    \end{frame}

    \begin{frame}{Gn 6.5--8 (TB)}
        \QUOTE{%
            %-----!j 92 -i12
            \ver{5}~Viu Jeová que era \RED{grande a maldade} do homem na terra e que \YEL{toda}
            a imaginação dos \RED{pensamentos} do seu coração era \RED{má} \YEL{continuamente}.
            %-----!j 92 -i12
            \ver{6}~Então, se arrependeu\footnote{\YEL{Nacham}: \BRI{suspirar}, \BRI{respirar
            com força} (Strong's Exhaustive Concordance).} Jeová de ter feito o homem na terra,
            e pesou-lhe em seu coração.
        }
    \end{frame}

    \begin{frame}{Gn 6.5--8 (TB)}
        \QUOTE{%
            %-----!j 92 -i12
            \ver{7}~Disse Jeová: Farei \RED{desaparecer da face da terra o homem que criei},
            desde o homem até o animal, até os répteis e até as aves do céu; porque me arrependo
            de os haver feito.
            %-----!j 92 -i12
            \ver{8}~\GRE{Porém Noé achou graça aos olhos de Jeová}.
        }
    \end{frame}

    %-------------------------------------------------------------------------------------------
    \subsection{Sumário}
    %-------------------------------------------------------------------------------------------

    \begin{frame}
        \par\noindent\hspace*{0.05\linewidth}%
        \begin{minipage}{0.9\linewidth}%
            \large%
            \begin{alertblock}{Sumário --- O Período de Consciência}
                \normalsize
                \begin{itemize}
                    \item<1-> Mostra quão trágico o ``\RED{conhecimento do (que é) bom e mau}'';
                    \item<1-> Idolatria, facções, invejas, ódios, homicídios; lascívia, soberba, vinganças;
                    \item<1-> Constatação: ``grande maldade \YEL{do homem}'', com
                    \item<1-> ``toda a imaginação dos pensamentos \YEL{do seu coração} má
                        continuamente''.
                    \item<1-> Tragédia: \RED{cada um determinando de si}, o que é \YEL{bom/mau}
                        e o que \GRE{fazer}\footnote{$\parallelsum$ ``Naqueles dias, \BRI{não
                        havia rei} ... \YEL{cada um} \GRE{fazia} o que \GRE{achava} mais
                        \YEL{reto}.'' -- Jz 21.25.}.
                    \item<1-> Conhecimento de Deus: \YEL{tradição oral} / \YEL{sacrifícios};
                    \item<1-> Cada um \RED{cuida de si}. / \ORA{Não se relata a formação de
                        nenhum governo}!
                    \item<1-> \GRE{Porém Noé achou graça aos olhos de Jeová}.
                \end{itemize}
            \end{alertblock}
        \end{minipage}%
    \end{frame}

%-----------------------------------------------------------------------------------------------
\end{document}
%-----------------------------------------------------------------------------------------------

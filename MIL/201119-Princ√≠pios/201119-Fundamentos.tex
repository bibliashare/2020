%-----------------------------------------------------------------------------------------------
\documentclass[12pt,aspectratio=169]{beamer}
%-----------------------------------------------------------------------------------------------
\usepackage{pslatex}
%-----------------------------------------------------------------------------------------------
\newcommand{\YA}{%
    \mbox{%
        Y\makebox[0pt][l]{\hspace{-0.178em}\raisebox{-0.00ex}{\scalebox{0.30}{E}}}%
        H\makebox[0pt][l]{\hspace{-0.010em}\raisebox{-0.00ex}{\scalebox{0.30}{O}}}%
        W\makebox[0pt][l]{\hspace{-0.245em}\raisebox{-0.00ex}{\scalebox{0.30}{A}}}%
        H%
    }%
}
%-----------------------------------------------------------------------------------------------
\newcommand{\ver}[1]{%
    \raisebox{0.50ex}{%
        \scalebox{1.1}{%
            \pmb{\textbf{\textcolor{BSpbg}{#1}}}%
        }%
    }%
}
%-----------------------------------------------------------------------------------------------
\newcommand{\QUOTE}[1]{%
    \par\noindent\hspace*{0.05\linewidth}%
    \begin{minipage}{0.9\linewidth}%
        \linespread{1.35}\large{#1}%
    \end{minipage}%
}
%-----------------------------------------------------------------------------------------------
\newcommand{\parallelsum}{\mathbin{\!/\mkern-5mu/\!}} % https://tex.stackexchange.com/a/37917
%-----------------------------------------------------------------------------------------------
\newcommand{\RED}[1]{{\textcolor{TXred}{#1}}}
\newcommand{\ORA}[1]{{\textcolor{TXred!50!TXyel}{#1}}}
\newcommand{\YEL}[1]{{\textcolor{TXyel}{#1}}}
\newcommand{\GRE}[1]{{\textcolor{TXgre}{#1}}}
\newcommand{\CYA}[1]{{\textcolor{TXcya}{#1}}}
\newcommand{\BLU}[1]{{\textcolor{TXblu}{#1}}}
\newcommand{\MAG}[1]{{\textcolor{TXmag}{#1}}}
\newcommand{\BRI}[1]{{\textcolor{BSpbg}{#1}}}   % Bright
%-----------------------------------------------------------------------------------------------
\usetheme{CambridgeUS}
\usefonttheme{serif}
\usecolortheme{BShare1}
%-----------------------------------------------------------------------------------------------
\title[Reino de Cristo -- Princípios]{Reino de Cristo}
\subtitle{Princípios}
\author{Bíblia Share}
%\institute{Bíblia Share}
\date[{\tiny\tt 19 de Novembro de 2020}]{{\scriptsize\tt%
    \includegraphics[height=6.0mm]{res/cc/by-nc-nd-88x31.pdf}\\[\smallskipamount]
    19 de Novembro de 2020
}}
%-----------------------------------------------------------------------------------------------
\begin{document}
%-----------------------------------------------------------------------------------------------
\logo{%
    \parbox{158mm}{% There's a 1mm gap on each side of the 160mm x 90mm slide logo line
    \mode<beamer>{%
        \hfill\includegraphics[height=9.0mm]{res/logo/BibliaShare.pdf}%
    }
    \mode<handout>{%
        \hfill\includegraphics[height=9.0mm]{res/logo/BibliaShare.pdf}%
    }
}}
%-----------------------------------------------------------------------------------------------
\begin{frame}
    \titlepage
\end{frame}
%-----------------------------------------------------------------------------------------------
\section{O Primeiro Adão}
%-----------------------------------------------------------------------------------------------

    \begin{frame}
        \par\noindent\hspace*{0.05\linewidth}%
        \begin{minipage}{0.9\linewidth}%
            \large%
            \begin{alertblock}{Tópico}
                O Primeiro Adão
            \end{alertblock}
        \end{minipage}%
    \end{frame}

    %-------------------------------------------------------------------------------------------
    \subsection{Imagem de Deus dominando o mundo material}
    %-------------------------------------------------------------------------------------------

    \begin{frame}{Gn 1.1,26,28 (TB)}
        \QUOTE{%
            %-----!j 92 -i12
            \ver{1}~No princípio, criou Deus o \CYA{céu} e a \YEL{terra}. [...]
            %-----!j 92 -i12
            \ver{26}~Disse também Deus: Façamos o \YEL{homem} \MAG{à nossa imagem}, conforme a
            \MAG{nossa semelhança}; \YEL{domine ele} sobre os peixes do mar, sobre as aves do
            céu, sobre os animais domésticos, \YEL{sobre toda a terra} e sobre todo réptil que
            se arrasta sobre a terra.
            %-----!j 92 -i12
            \ver{28}~Deus os abençoou e lhes disse: Frutificai, multiplicai-vos, \YEL{enchei a
            terra} e \YEL{sujeitai-a}; \YEL{dominai} sobre os peixes do mar, sobre as aves do
            céu e sobre todos os animais que se arrastam sobre a terra.
        }
    \end{frame}

    \begin{frame}{Gn 2.15,19 (TB)}
        \QUOTE{%
            %-----!j 92 -i12
            \ver{15}~Tomou, pois, o \MAG{Senhor Deus} ao \GRE{homem} e o colocou no jardim do
            Éden \YEL{para o cultivar e o guardar}.
            %-----!j 92 -i12
            \ver{19}~Havendo, pois, o \MAG{Senhor Deus} formado da terra todos os animais do
            campo e todas as aves dos céus, trouxe-os ao \GRE{homem}, para ver como \YEL{este
            lhes chamaria}; e o nome que o homem desse a todos os seres viventes, \YEL{esse
            seria o nome deles}.
        }
    \end{frame}

    \begin{frame}{Ez 28.13a,14--15 (ARA), 16b (TB)}
        \QUOTE{%
            %-----!j 92 -i12
            \ver{13}~\YEL{Estavas no Éden}, jardim de Deus; [...]
            %-----!j 92 -i12
            \ver{14}~Tu eras \YEL{querubim} da \YEL{guarda} ungido, e te \YEL{estabeleci};
            permanecias no monte santo de Deus, no brilho das pedras andavas.
            %-----!j 92 -i12
            \ver{15}~\YEL{Perfeito} eras nos teus caminhos, desde o dia em que foste
            \YEL{criado} até que se achou \RED{iniquidade} em ti.
            %-----!j 92 -i12
            \ver{16}~[...] e \RED{pecaste}; portanto te \RED{lancei, profanado}, do monte de
            Deus, e te exterminei, ó \YEL{querubim} cobridor, [...]
        }
    \end{frame}

    \begin{frame}{Jo 8.34,44; 10.10 (TB)}
        \QUOTE{%
            %-----!j 92 -i12
            \ver{8.34}~Replicou-lhes Jesus: Em verdade, em verdade vos digo: \YEL{todo} o que
            comete \RED{pecado} é \RED{escravo do pecado}.
            %-----!j 92 -i12
            \ver{8.44}~Vós sois filhos do \RED{Diabo} e tendes vontade de cumprir os
            \RED{desejos} de vosso pai. Ele era \RED{homicida desde o princípio} e não
            permaneceu na verdade, porque não há nele verdade. Quando ele diz uma mentira, fala
            do que lhe é próprio, porque é \RED{mentiroso} e o pai da mentira.
            %-----!j 92 -i12
            \ver{10.10}~O ladrão não vem senão para \RED{furtar, matar e destruir}; eu vim para
            que elas tenham vida e a tenham em abundância.
        }
    \end{frame}

    %-------------------------------------------------------------------------------------------
    \subsection{Queda, Maldição e Promessas de Restauração}
    %-------------------------------------------------------------------------------------------

    \begin{frame}{Gn 2.17; 3.6,7 (TB)}
        \QUOTE{%
            %-----!j 92 -i12
            \ver{2.17}~mas da árvore do \YEL{conhecimento} do \YEL{bem} e do \YEL{mal} não
            comerás; porque, no dia em que dela comeres, certamente morrerás.
            %-----!j 92 -i12
            \ver{3.6}~Vendo a mulher que a árvore era boa para se comer, agradável aos olhos e
            árvore \RED{desejável para dar entendimento}, tomou-lhe do fruto e \RED{comeu} e deu
            também ao marido, e ele \RED{comeu}.
            %-----!j 92 -i12
            \ver{3.7}~Abriram-se, então, os olhos de ambos; e, \YEL{percebendo} que estavam nus,
            \YEL{coseram} folhas de figueira \YEL{e fizeram} cintas para si.
        }
    \end{frame}

    \begin{frame}{Gn 3.17--19 (TB)}
        \QUOTE{%
            %-----!j 92 -i12
            \ver{17}~E a Adão disse: Porque escutaste a voz de tua mulher e comeste da árvore de
            que te ordenei que não comesses, \RED{maldita é a terra por tua causa}; em fadiga
            tirarás dela o sustento todos os \RED{dias da tua vida}.
            %-----!j 92 -i12
            \ver{18}~Ela te produzirá também \RED{espinhos e abrolhos}, e comerás as ervas do
            campo.
            %-----!j 92 -i12
            \ver{19}~No \RED{suor do teu rosto comerás o teu pão}, até que te tornes à terra,
            pois dela foste tomado: porquanto \RED{tu és pó e em pó te hás de tornar}.
        }
    \end{frame}

    \begin{frame}{Gn 3.14 (TB) 15 (ARA)}
        \QUOTE{%
            %-----!j 92 -i12
            \ver{14}~Então, disse Deus Jeová à \RED{serpente}: Porquanto assim o fizeste,
            \RED{maldita és} tu dentre todos os animais domésticos e dentre todos os animais do
            campo; sobre o teu ventre andarás de rastos, o pó comerás todos os \RED{dias da tua
            vida}.
            %-----!j 92 -i12
            \ver{15}~Porei inimizade entre \RED{ti} e a \YEL{mulher}, entre a \RED{tua
            descendência} e \YEL{o seu descendente}.  / \YEL{Este te ferirá a cabeça}, e \RED{tu
            lhe ferirás o calcanhar}.
        }
    \end{frame}

    %-------------------------------------------------------------------------------------------
    \subsection{Sumário}
    %-------------------------------------------------------------------------------------------

    \begin{frame}
        \par\noindent\hspace*{0.05\linewidth}%
        \begin{minipage}{0.9\linewidth}%
            \large%
            \begin{alertblock}{Sumário --- O Primeiro Adão (1 de 2)}
                \normalsize
                \begin{itemize}
                    \item<1-> \ORA{Primeiro} homem \ORA{Adão}, feito à \ORA{imagem de Deus},
                        \ORA{dominando} a \ORA{terra};
                    \item<1-> Inicialmente \ORA{sozinho}, no \ORA{jardim};
                    \item<1-> Depois coletivamente (\ORA{corpo}), por \ORA{toda a terra};
                    \item<1-> \ORA{Deus} propunha, o \ORA{homem} fazia, e o fazia com
                        \ORA{liberdade de fato}\footnote{\uncover<1->{Conf.~``o nome que \YEL{o
                        homem desse} (aos) seres ... \YEL{esse seria o nome deles}.'' -- Gn
                        2.19.}}.
                    \item<2-> O \CYA{último Adão}, Cristo Jesus encarnado, é a \CYA{expressão
                        exata de Deus}\footnote{\uncover<2->{Conf.~Hb 1.3; ``\GRE{imagem do Deus
                        invisível}'' -- Cl 1.15; ``\GRE{eu e o Pai somos um}'' -- Jo 10.30}};
                    \item<2-> Inicialmente \CYA{sozinho}, grão de \CYA{trigo}, caído na terra;
                    \item<2-> Depois coletivamente (\CYA{corpo}), no erguer da cruz, \CYA{atraiu
                        a todos};
                    \item<2-> Mesmo podendo, \CYA{não age de si
                        mesmo}\footnote{\uncover<2->{Conf.~Jo 5.30,31; ``\GRE{nada faço por mim
                        mesmo; mas como o Pai me ensinou}'' -- 8.28}}, e dá \CYA{vida
                        abundante}.
                \end{itemize}
            \end{alertblock}
        \end{minipage}%
    \end{frame}

    \begin{frame}
        \par\noindent\hspace*{0.05\linewidth}%
        \begin{minipage}{0.9\linewidth}%
            \large%
            \begin{alertblock}{Sumário --- O Primeiro Adão (2 de 2)}
                \normalsize
                \begin{itemize}
                    \item<1-> \RED{Maldição} sobre \RED{toda a terra} \ORA{por causa} de
                        Adão\footnote{Mostrando realmente sua então posição de dominador.};
                    \item<1-> \RED{Sustento difícil}: Suor do rosto; espinhos e abrolhos;
                    \item<1-> Duração da vida do homem (e animais) agora \RED{contada em dias};
                    \item<1-> Promessa da \RED{morte} (retorno ao pó);
                    \item<1-> \RED{Inimizade} entre a \RED{descendência da
                        serpente}\footnote{Conf.~Jo 8.44: ``\GRE{Vós sois do diabo, que é vosso
                        pai, e quereis satisfazer-lhe os desejos.}''} e \MAG{o descendente da
                        mulher};
                    \item<1-> Promessa d\MAG{o descendente da mulher} ferir a cabeça da
                        serpente;
                    \item<1-> Confirmação que o homem adquiriu o \RED{conhecimento do (que é)
                        bom/mau}\footnote{Conf.~Gn 3.22: ``\GRE{Eis que o homem se tornou como
                        um de nós, conhecedor do bem e do mal;}''};
                    \item<1-> Homem \RED{lançado fora} do jardim, \RED{sem acesso} à árvore da
                        vida.
                \end{itemize}
            \end{alertblock}
        \end{minipage}%
    \end{frame}

%-----------------------------------------------------------------------------------------------
\section{Período de Consiência}
%-----------------------------------------------------------------------------------------------

    \begin{frame}
        \par\noindent\hspace*{0.05\linewidth}%
        \begin{minipage}{0.9\linewidth}%
            \large%
            \begin{alertblock}{Tópico}
                Período de Consiência
            \end{alertblock}
        \end{minipage}%
    \end{frame}

    %-------------------------------------------------------------------------------------------
    \subsection{Violência e Caos da Humanidade Caída}
    %-------------------------------------------------------------------------------------------

    \begin{frame}{Gn 4.1,2,8 (TB)}
        \QUOTE{%
            %-----!j 92 -i12
            \ver{1}~O homem conheceu a Eva, sua mulher; ela concebeu e, dando à luz a
            \YEL{Caim}\footnote{\uncover<2->{\YEL{Caim}: `uma lança,' `ferreiro,' (Cambridge
            Bible) ou `possuir,' `estabelecer' --- talvez pensando que fosse a prometida
            semente?  (Henry's).}}, disse: Adquiri um homem com o auxílio de Jeová.
            %-----!j 92 -i12
            \ver{2}~Tornou a dar à luz a um filho, a
            \YEL{Abel}\footnote{\uncover<3->{\YEL{Abel}: `vaidade,' `sopro' --- talvez pensando
            ser inútil ter outro filho? (Henry's).}}, seu irmão. Abel foi pastor de ovelhas, mas
            Caim foi lavrador da terra.
            %-----!j 92 -i12
            \ver{8}~Sucedeu, pois, que, estando eles no campo, se levantou Caim contra seu irmão
            Abel \RED{e o matou}.
        }
    \end{frame}

    \begin{frame}{Gn 4.23,24 (TB)}
        \QUOTE{%
            %-----!j 92 -i12
            \ver{23}~Disse Lameque\footnote{Lameque: A quinta geração de Caim} à suas mulheres:
            Ada e Zilá, ouvi a minha voz; Vós, mulheres de Lameque, escutai as minhas palavras:
            pois \RED{matei um homem, porque me feriu}; e um \RED{mancebo, porque me pisou}.
            %-----!j 92 -i12
            \ver{24}~Se por Caim tomar-se-á vingança sete vezes, com certeza, por Lameque o será
            setenta e sete vezes\footnote{Pelo tempo típico entre gerações, no Cap.~5, a palavra
            de Deus é recitada $\approx 500$ anos após o fato, indicando forte tradição oral.}.
        }
    \end{frame}

    \begin{frame}{Gn 5.28,29 (TB)}
        \QUOTE{%
            %-----!j 92 -i12
            \ver{28}~Lameque\footnote{Este é descendente de Sete, não de Caim. Seguindo a
            genealogia do Cap.~5, a frase do v.~29 vem \BRI{$1056$ anos após a criação}, e
            \BRI{$126$ anos após a morte de Adão} --- tradição oral.} viveu cento e oitenta e
            dois anos e gerou um filho,
            %-----!j 92 -i12
            \ver{29}~a quem chamou Noé, dizendo: Este nos dará \YEL{descanso das nossas obras e
            do trabalho} das nossas mãos, \YEL{os quais vêm da terra que Jeová amaldiçoou}.
        }
    \end{frame}

    \begin{frame}{Gn 6.5--8 (TB)}
        \QUOTE{%
            %-----!j 92 -i12
            \ver{5}~Viu Jeová que era \RED{grande a maldade} do homem na terra e que \YEL{toda}
            a imaginação dos \RED{pensamentos} do seu coração era \RED{má} \YEL{continuamente}.
            %-----!j 92 -i12
            \ver{6}~Então, se arrependeu\footnote{\YEL{Nacham}: \BRI{suspirar}, \BRI{respirar
            com força} (Strong's Exhaustive Concordance).} Jeová de ter feito o homem na terra,
            e pesou-lhe em seu coração.
        }
    \end{frame}

    \begin{frame}{Gn 6.5--8 (TB)}
        \QUOTE{%
            %-----!j 92 -i12
            \ver{7}~Disse Jeová: Farei \RED{desaparecer da face da terra o homem que criei},
            desde o homem até o animal, até os répteis e até as aves do céu; porque me arrependo
            de os haver feito.
            %-----!j 92 -i12
            \ver{8}~\GRE{Porém Noé achou graça aos olhos de Jeová}.
        }
    \end{frame}

    %-------------------------------------------------------------------------------------------
    \subsection{Sumário}
    %-------------------------------------------------------------------------------------------

    \begin{frame}
        \par\noindent\hspace*{0.05\linewidth}%
        \begin{minipage}{0.9\linewidth}%
            \large%
            \begin{alertblock}{Sumário --- O Período de Consciência}
                \normalsize
                \begin{itemize}
                    \item<1-> Mostra quão trágico o ``\RED{conhecimento do (que é) bom e mau}'';
                    \item<1-> Idolatria, facções, invejas, ódios, homicídios; lascívia, soberba, vinganças;
                    \item<1-> Constatação: ``grande maldade \YEL{do homem}'', com
                    \item<1-> ``toda a imaginação dos pensamentos \YEL{do seu coração} má
                        continuamente''.
                    \item<2-> Tragédia: \RED{cada um determinando de si}, o que é \YEL{bom/mau}
                        e o que \GRE{fazer}\footnote{$\parallelsum$ ``Naqueles dias, \BRI{não
                        havia rei} ... \YEL{cada um} \GRE{fazia} o que \GRE{achava} mais
                        \YEL{reto}.'' -- Jz 21.25.}.
                    \item<2-> Conhecimento de Deus: \YEL{tradição oral} / \YEL{sacrifícios};
                    \item<2-> Cada um \RED{cuida de si}. / \ORA{Não se relata a formação de
                        nenhum governo}!
                    \item<2-> \GRE{Porém Noé achou graça aos olhos de Jeová}.
                \end{itemize}
            \end{alertblock}
        \end{minipage}%
    \end{frame}

%-----------------------------------------------------------------------------------------------
\section{Aliança com Noé (Governo Humano)}
%-----------------------------------------------------------------------------------------------

    \begin{frame}
        \par\noindent\hspace*{0.05\linewidth}%
        \begin{minipage}{0.9\linewidth}%
            \large%
            \begin{alertblock}{Tópico}
                Aliança com Noé (Governo Humano)
            \end{alertblock}
        \end{minipage}%
    \end{frame}

    %-------------------------------------------------------------------------------------------
    \subsection{O Dilúvio}
    %-------------------------------------------------------------------------------------------

    \begin{frame}{Gn 6.13,14,17 (TB)}
        \QUOTE{%
            %-----!j 92 -i12
            \ver{13}~Disse Deus a Noé: Hei resolvido dar cabo de toda a carne, porque a terra
            está cheia da \RED{violência dos homens}; eis que eu os farei perecer juntamente com
            a terra.
            %-----!j 92 -i12
            \ver{14}~Faze para ti uma \GRE{arca} de madeira de gofer; compartimentos farás na
            arca e untá-la-ás com betume por dentro e por fora.
            %-----!j 92 -i12
            \ver{17}~Eis que eu vou trazer o \YEL{dilúvio} de águas sobre a terra, para fazer
            perecer de debaixo do céu toda a carne, em que há o espírito de vida; \RED{tudo} o
            que há na terra \RED{morrerá}.
        }
    \end{frame}

    \begin{frame}{Gn 6.18; 7.5,7,11 (TB)}
        \QUOTE{%
            %-----!j 92 -i12
            \ver{6.18}~Porém, \YEL{contigo estabelecerei a minha aliança}; entrarás na arca,
            \GRE{tu} com \GRE{teus filhos}, \GRE{tua mulher} e as \GRE{mulheres de teus filhos}.
            %-----!j 92 -i12
            \ver{7.5}~Fez Noé segundo tudo o que Jeová lhe ordenara. [...]
            %-----!j 92 -i12
            \ver{7.7}~Entrou na arca Noé com seus filhos, sua mulher e as mulheres de seus
            filhos, [...]
            %-----!j 92 -i12
            \ver{7.11}~No \YEL{ano} seiscentos da vida de Noé, no \YEL{mês} segundo, no
            \YEL{dia} dezessete do mês, \YEL{nesse dia}, romperam-se as fontes do grande abismo,
            e abriram-se as janelas do céu.
        }
    \end{frame}

    \begin{frame}{Gn 7.16,19,22; Ml 3.18 (TB)}
        \QUOTE{%
            %-----!j 92 -i12
            \ver{Gn 7.16}~Os que entraram, entraram macho e fêmea de toda a carne, como Deus lhe
            ordenara; \YEL{e Jeová o fechou dentro}.
            %-----!j 92 -i12
            \ver{7.19}~As águas prevaleceram excessivamente sobre a terra, e \YEL{todos} os
            altos \YEL{montes} que havia debaixo do céu foram \YEL{cobertos}.
            %-----!j 92 -i12
            \ver{7.22}~\YEL{tudo} o que tinha o \GRE{fôlego do espírito de vida em seus
            narizes}, tudo o que havia na terra seca \RED{morreu}.
            %-----!j 92 -i12
            \ver{Ml 3.18}~Então, \YEL{voltarei} e \YEL{discernirei} entre o \GRE{justo} e o
            \RED{ímpio}, entre o que serve a Deus e o que não o serve.
        }
    \end{frame}

    \begin{frame}{Gn 8.13--16 (TB)}
        \QUOTE{%
            %-----!j 92 -i12
            \ver{13}~No \YEL{ano} seiscentos e um, no primeiro \YEL{mês}, no primeiro \YEL{dia}
            do mês, secaram-se as águas de cima da terra e, tirando a coberta da arca, olhou
            Noé, e eis que a face da terra estava enxuta.
            %-----!j 92 -i12
            \ver{14}~No segundo mês, aos vinte e sete dias do mês, a terra estava \YEL{seca}.
            %-----!j 92 -i12
            \ver{15}~Então, disse Deus a Noé:
            %-----!j 92 -i12
            \ver{16}~\GRE{Sai da arca, tu com tua mulher, teus filhos e as mulheres de teus
            filhos}.
            %-----!j 92 -i12
            \ver{20}~Edificou Noé um altar a Jeová; [...] e \ORA{ofereceu holocaustos} sobre o
            altar.
        }
    \end{frame}

    %-------------------------------------------------------------------------------------------
    \subsection{Aliança com Noé}
    %-------------------------------------------------------------------------------------------

    \begin{frame}{Gn 9.1--4 (TB)}
        \QUOTE{%
            %-----!j 92 -i12
            \ver{1}~Abençoou Deus a Noé e a seus filhos e lhes disse: \YEL{Frutificai,
            multiplicai-vos e enchei a terra}.
            %-----!j 92 -i12
            \ver{2}~Terá \ORA{medo e pavor de vós todo o animal da terra e toda a ave do céu};
            nas vossas mãos serão eles entregues juntamente com tudo o que se move sobre a terra
            e com todos os peixes do mar.
            %-----!j 92 -i12
            \ver{3}~Tudo o que se move e vive vos servirá de mantimento; \YEL{como a erva verde,
            tudo vos tenho dado a vós}.
            %-----!j 92 -i12
            \ver{4}~A carne, porém, com sua vida, isto é, com seu \RED{sangue, não comereis}.
        }
    \end{frame}

    \begin{frame}{Gn 9.5,6 (TB)}
        \QUOTE{%
            %-----!j 92 -i12
            \ver{5}~\YEL{Certamente, requererei o vosso sangue}, o sangue das vossas vidas;
            \ORA{da mão de todo o animal, o requererei}; e, \RED{da mão do homem}, sim, da mão
            do irmão de cada um, requererei a vida do homem.
            %-----!j 92 -i12
            \ver{6}~\YEL{Se alguém derramar o sangue do homem}, \GRE{pelo homem será derramado o
            seu sangue}; porque \CYA{o homem foi feito à imagem de Deus}.
        }
    \end{frame}

    \begin{frame}{Gn 9.12,13 (TB)}
        \QUOTE{%
            %-----!j 92 -i12
            \ver{12}~Disse Deus: Este é o \YEL{sinal da aliança} que faço entre mim e vós e todo
            o animal vivente que está convosco, para \BRI{perpétuas gerações}:
            %-----!j 92 -i12
            \ver{13}~\YEL{o meu arco} tenho posto nas nuvens, e será ele por sinal de uma
            \YEL{aliança entre mim e a terra}.
        }
    \end{frame}

    \begin{frame}{Aliança eterna: \BRI{Ainda vigente!} -- Rm 13.1--5,7 (TB)}
        \QUOTE{%
            %-----!j 92 -i12
            \ver{1}~Todo homem esteja \YEL{sujeito} às autoridades superiores. Pois \YEL{não há
            autoridade que não venha de Deus}; e as que há têm sido \YEL{ordenadas por Deus}.
            %-----!j 92 -i12
            \ver{2}~De modo que aquele que se \RED{opõe} à autoridade \RED{resiste à ordenação
            de Deus}; e os que resistem trarão sobre si \RED{condenação}.
            %-----!j 92 -i12
            \ver{3}~Os magistrados não são para temor quando se faz o que é bom, mas quando se
            faz o que é mau. Queres tu não temer a autoridade? \YEL{Faze o bem} e terás louvor
            dela,
        }
    \end{frame}

    \begin{frame}{Aliança eterna: Ainda vigente! -- Rm 13.1--5,7 (TB)}
        \QUOTE{%
            %-----!j 92 -i12
            \ver{4}~porque a autoridade é \YEL{ministro de Deus} para o teu bem. Mas, se fizeres
            o mal, teme; porque ela não traz debalde a \MAG{espada}, pois é \MAG{ministro de
            Deus, vingador} para exercer ira naquele que pratica o mal.
            %-----!j 92 -i12
            \ver{5}~É necessário que \YEL{lhe estejais sujeitos} não somente por causa da ira,
            mas também por causa da \YEL{consciência}.
            %-----!j 92 -i12
            \ver{7}~\YEL{Pagai a todos o que lhes é devido}: a quem tributo, \GRE{tributo}; a
            quem imposto, \GRE{imposto}; a quem temor, \GRE{temor}; a quem honra, \GRE{honra}.
        }
    \end{frame}

    \begin{frame}{Is 24.5; Ap 4.2,3 (TB)}
        \QUOTE{%
            %-----!j 92 -i12
            \ver{Is 24.5}~Na verdade, a terra está contaminada por causa dos seus moradores,
            porquanto transgridem as leis, violam os estatutos e \YEL{quebram a aliança eterna}.
            %-----!j 92 -i12
            \ver{Ap 4.2}~Imediatamente, fui arrebatado pelo Espírito. Eis havia um \YEL{trono
            posto no céu}, e, sobre o trono, um sentado;
            %-----!j 92 -i12
            \ver{Ap 4.3}~e aquele que estava sentado era, pelo que parecia, semelhante a uma
            pedra de jaspe e de sardônio; e \YEL{havia ao redor do trono um arco-íris}
            semelhante, pelo que parecia, à \GRE{esmeralda}.
        }
    \end{frame}

    {\usebackgroundtemplate{\parbox{\paperwidth}{
        \vspace*{1sp}\centering\includegraphics[height=\paperheight]{{fig/emerald.jpg}}
    }}\frame[plain]{%
        \vspace*{72mm}\color{white}\scriptsize\bf{Rough emerald crystals from Panjshir
        Valley Afghanistan by Paweł Maliszczak}
    }\usebackgroundtemplate{\mbox{~}}}

    %-------------------------------------------------------------------------------------------
    \subsection{Torre de Babel e Surgimento das Nações}
    %-------------------------------------------------------------------------------------------

    \begin{frame}{Gn 11.4--9 (TB)}
        \QUOTE{%
            %-----!j 92 -i12
            \ver{4}~E disseram: Vinde, edifiquemos para nós uma cidade e uma torre cujo cume
            chegue até o céu e façamo-nos um nome; para que \RED{não sejamos espalhados} sobre a
            face de toda a terra.
            %-----!j 92 -i12
            \ver{5}~Porém, desceu Jeová para ver a cidade e a torre que os filhos dos homens
            edificavam.
            %-----!j 92 -i12
            \ver{6}~Disse Jeová: Eis que o povo é um só, e todos eles têm uma só linguagem.
            \ORA{Isso é o que começam} a fazer: agora, nada lhes será vedado de quanto
            \YEL{intentam fazer}.
        }
    \end{frame}

    \begin{frame}{Gn 11.4--9 (TB)}
        \QUOTE{%
            %-----!j 92 -i12
            \ver{7}~Vinde, desçamos e confundamos ali a sua linguagem, para que não entendam a
            linguagem um do outro.
            %-----!j 92 -i12
            \ver{8}~Assim Jeová os \YEL{espalhou dali} sobre a face de toda a terra; e cessaram
            de edificar a cidade.
            %-----!j 92 -i12
            \ver{9}~Por isso, se chamou o seu nome \YEL{Babel}, porquanto ali \YEL{confundiu}
            Jeová a linguagem de toda a terra; e dali os \YEL{espalhou} sobre a face de toda a
            terra.
        }
    \end{frame}

%-----------------------------------------------------------------------------------------------
\section{Alianças com Abrão (Patriarca de Israel)}
%-----------------------------------------------------------------------------------------------

    \begin{frame}
        \par\noindent\hspace*{0.05\linewidth}%
        \begin{minipage}{0.9\linewidth}%
            \large%
            \begin{alertblock}{Tópico}
                Alianças com Abrão (Patriarca de Israel)
            \end{alertblock}
        \end{minipage}%
    \end{frame}

    %-------------------------------------------------------------------------------------------
    \subsection{Chamado e Promessas \BRI{Incondicionais}}
    %-------------------------------------------------------------------------------------------

    \begin{frame}{Gn 12.1--3 (TB)}
        \QUOTE{%
            %-----!j 92 -i12
            \ver{1}~Ora, disse Jeová a Abrão: Sai-te da tua terra, da tua parentela e da casa de
            teu pai para a terra que te mostrarei;
            %-----!j 92 -i12
            \ver{2}~farei de ti uma \YEL{grande nação}, e te \GRE{abençoarei}, e
            \CYA{engrandecerei} o teu nome.  \GRE{Sê tu uma bênção}.
            %-----!j 92 -i12
            \ver{3}~\BLU{Abençoarei os que te abençoarem} e \RED{amaldiçoarei aquele que te
            amaldiçoar}; por meio de ti, serão \MAG{benditas todas as famílias da terra}.
        }
    \end{frame}

    \begin{frame}{Mt 1.1; Gl 3.1,8 (TB)}
        \QUOTE{%
            %-----!j 92 -i12
            \ver{Mt 1.1}~Livro da geração de \MAG{Jesus Cristo}, filho de Davi, \YEL{filho de
            Abraão}.
            \\[\medskipamount]
            %-----!j 92 -i12
            \ver{Gl 3.1}~Ó insensatos gálatas! Quem vos fascinou a vós, ante cujos olhos foi
            representado \MAG{Jesus Cristo} como crucificado?
            %-----!j 92 -i12
            \ver{Gl 3.8}~A \YEL{Escritura}, \YEL{prevendo} que Deus justificaria os gentios pela
            fé, \YEL{de antemão, anunciou} as boas-novas a Abraão: \MAG{Em ti serão
            bem-aventuradas todas as nações}.
        }
    \end{frame}

    \begin{frame}{Promessas \YEL{Incondicionais} de \BRI{Gn 12}:}
        \begin{itemize}
            \item<1-> Abrão será uma \YEL{grande nação} (Israel);
            \item<2-> Abrão (Israel) será \YEL{abençoado por Deus};
            \item<3-> Abrão (Israel) terá seu \YEL{nome engrandecido};
            \item<4-> Abrão (Israel) será uma \YEL{bênção};
            \item<5-> Deus \GRE{abençoará quem abençoar} a Abrão (Israel);
            \item<6-> Deus \RED{amaldiçoará quem amaldiçoar} a Abrão (Israel);
            \item<7-> Em Abrão (Jesus) serão \MAG{benditas todas as famílias da terra}!
        \end{itemize}
    \end{frame}

    %-------------------------------------------------------------------------------------------
    \subsection{Promessas \BRI{Incondicionais} de Gn 15}
    %-------------------------------------------------------------------------------------------

    \begin{frame}{Gn 15 (TB)}
        \QUOTE{%
            %-----!j 92 -i12
            \ver{1}~Depois destas coisas, veio a palavra de Jeová a Abrão, numa visão, dizendo:
            Não temas, Abrão; eu sou teu \YEL{escudo}, a tua \YEL{recompensa} será infinitamente
            grande.
            %-----!j 92 -i12
            \ver{2}~Respondeu Abrão: Senhor Jeová, que me darás, visto que morro sem filhos, e o
            herdeiro da minha casa é Eliézer de Damasco?
            %-----!j 92 -i12
            \ver{3}~Acrescentou Abrão: Eis que a mim não me tens dado filhos, e um escravo vai
            ser o meu herdeiro.
        }
    \end{frame}

    \begin{frame}{Gn 15 (TB)}
        \QUOTE{%
            %-----!j 92 -i12
            \ver{4}~Veio-lhe a palavra de Jeová: Este não será o teu herdeiro; porém aquele que
            será \YEL{gerado de ti} será o teu herdeiro.
            %-----!j 92 -i12
            \ver{5}~Fez-lhe sair para fora e disse: Olha para o céu e \CYA{conta as estrelas},
            se as poderes contar; e disse-lhe: \YEL{Assim será a tua semente}.
            %-----!j 92 -i12
            \ver{6}~\GRE{Creu} Abrão em Jeová, que \GRE{lhe imputou isto como justiça}.
        }
    \end{frame}

    \begin{frame}{Gn 15 (TB)}
        \QUOTE{%
            %-----!j 92 -i12
            \ver{7}~Disse-lhe mais: Eu sou Jeová, que te fiz sair de Ur dos caldeus, a fim de te
            dar \YEL{esta terra em herança}.
            %-----!j 92 -i12
            \ver{8}~Perguntou-lhe Abrão: Ó Senhor Jeová, \YEL{como saberei} que a hei de herdar?
            %-----!j 92 -i12
            \ver{9}~Respondeu-lhe: \YEL{Toma-me uma novilha} de três anos, e uma cabra de três
            anos, e um carneiro de três anos, e uma rola, e um pombinho.
        }
    \end{frame}

    \begin{frame}{Gn 15 (TB)}
        \QUOTE{%
            %-----!j 92 -i12
            \ver{10}~Ele, tomando todos esses animais, os partiu pelo meio e pôs cada metade em
            frente da outra; mas as aves não partiu.
            %-----!j 92 -i12
            \ver{11}~As aves de rapina desciam sobre os cadáveres, porém Abrão as enxotava.
        }
    \end{frame}

    \begin{frame}{Gn 15 (TB)}
        \QUOTE{%
            %-----!j 92 -i12
            \ver{12}~Quando o sol ia a entrar, caiu um \ORA{profundo sono} sobre Abrão; eis que
            lhe sobreveio um \RED{horror de grandes trevas}.
            %-----!j 92 -i12
            \ver{13}~E lhe foi dito: Sabe, \YEL{com certeza}, que a \RED{tua semente será
            peregrina} em \RED{terra alheia}, e será \RED{reduzida à escravidão}, e será
            \RED{aflita} por quatrocentos anos.
        }
    \end{frame}

    \begin{frame}{Gn 15 (TB)}
        \QUOTE{%
            %-----!j 92 -i12
            \ver{14}~Sabe, também, que eu hei de \RED{julgar} a nação a que têm de servir; e,
            depois, \YEL{sairão} com grandes riquezas.
            %-----!j 92 -i12
            \ver{15}~Tu, porém, \YEL{irás em paz} para teus pais; serás sepultado numa boa
            \YEL{velhice}.
            %-----!j 92 -i12
            \ver{16}~Na quarta geração, voltarão para cá, \YEL{porque} a \RED{medida da
            iniquidade dos amorreus ainda não está cheia}.
        }
    \end{frame}

    \begin{frame}{Gn 15 (TB)}
        \QUOTE{%
            %-----!j 92 -i12
            \ver{17}~Quando o sol já estava posto, e era escuro, um \YEL{fogo fumegante e uma
            tocha de fogo passaram por entre aquelas metades}.
            %-----!j 92 -i12
            \ver{18}~Naquele dia, \YEL{fez Jeová uma aliança com Abrão}, dizendo: \GRE{À tua
            semente tenho dado esta terra, desde o rio do Egito até o grande rio, o rio
            Eufrates}:
            %-----!j 92 -i12
            \ver{19}~o queneu, o quenezeu, o cadmoneu,
            %-----!j 92 -i12
            \ver{20}~o heteu, o ferezeu, os refains,
            %-----!j 92 -i12
            \ver{21}~o amorreu, o cananeu, o girgaseu e o jebuseu.
        }
    \end{frame}

    \begin{frame}{Promessas \YEL{Incondicionais} de \BRI{Gn 15}:}
        \begin{itemize}
            \item<1-> \BRI{\YA} é o \YEL{escudo} de Abrão (Israel);
            \item<2-> \BRI{\YA} é a \YEL{recompensa} $\infty$ de Abrão (Israel);
            \item<3-> Abrão terá seu herdeiro (descendente) \YEL{gerado dele};
            \item<4-> A semente de Abrão (Israel) será \YEL{numerosa} como as \CYA{estrelas};
            \item<5-> Abrão (Israel) \YEL{herdará} a \YEL{terra};
            \item<6-> Abrão morrerá em \YEL{paz} após uma \YEL{boa velhice};
            \item<7-> Sua quarta geração \YEL{voltará} para a \YEL{terra};
            \item<8-> \YEL{A Israel foi dado a terra, desde o rio do Egito até o rio Eufrates}.
        \end{itemize}
    \end{frame}

%-----------------------------------------------------------------------------------------------
\end{document}
%-----------------------------------------------------------------------------------------------
